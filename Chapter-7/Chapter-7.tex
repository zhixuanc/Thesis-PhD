\chapter{Volcanic Ash Transportation and Dispersal Simulation (VATD)} \label{chapter:ash-transportation}

\section{Introduction}

VATD models require initial conditions that describe an initial locus of ash that is then transported.
Most VATD models use a line source of some prescribed shape calibrated against an empirical expression for the height-MER relation. %These are written as functions of several parameters including key global descriptors of the volcanic plumes.
Such empirical vertical ash distributions may not be good representations of actual vertical ash distributions, which usually vary from case to case and have complex dependencies on eruption source parameters and atmospherics conditions.
The 3D plume simulation software (Plume-SPH) presented in this thesis provides an alternative way to create initial ash cloud without any assumption regarding plume shape. 
By eliminating assumed behavior associated with the semiempirical plume shape expressions and creating initial condition based on 3D plume simulation VATDs can  predict ash transport better.
We present in this chapter technical details regarding the creation of initial conditions based on output of Plume-SPH and integrating Plume-SPH with one VATDs.
The importance of initial condition is shown in sensitivity analyses which prove that volcanic ash transportation simulation is much more sensitive to initial condition than all other input parameters.
Case study of Pinatubo eruption demonstrates that this new method can improve accuracy of ash transportation forecast significantly with more realistic initial condition.
%These key global descriptors, such as maximum height, are usually assigned based on parameter calibration.    

\section{Background}

\subsection{Volcanic cloud forecast}

The fine-grained fraction of tephra (volcanic ash) can be widely dispersed and can lead to a degradation of air quality and pose threats to aviation. Identification of volcanic ash helps schedule the flights accordingly to avoid flying through area where ash presents. Numerical estimation of ash distribution using known wind fields is necessary if we are to accurately predict ash cloud evolution. Numerous VATDs have been developed by both civil and military aviation or meteorological agencies to provide forecasts of ash cloud motion \citep{witham2007comparison}. Most VATD models use a line source of some prescribed shape calibrated against an empirical expression for the height-MER relation. The empirical expressions are written as functions of several parameters including key global descriptors of the volcanic plumes. It is a common practice to pick up the ``best match" key global descriptors by parameter calibration. Often, 1D plume models are used to provide the key global descriptors.

No matter which method is adopted, these key global descriptors, define the initial condition for VATDs, either in forms of line source term or an initial ash cloud. Line source term only distribute ash particles vertically on a line. Sometimes, an extra step is adopted to spread ash particles of line source horizontally resulting in an initial ash cloud in 3D space.
The horizontal spread also depends on empirical expression. For example, PUFF spreads particle of line source uniformally in the horizontal direction.
That is to say, both vertical distribution and horizontal distribution, if applicable, are based on empirical expressions.
Considering the complexities of volcano eruption, the actual ash distribution in initial ash clouds should vary from case to case. It is difficult to have one general expression that suitable for all cases. As has been reported and will be confirmed in this chapter, initial condition has significant effects on simulation of volcanic ash transportation. Initial condition created based on assumed plume shape might be far away from realistic and impair the accuracy of ash transportation forecast. In this chapter, we investigate another way of creating the initial ash clouds without assumed plume shape based on outputs of Plume-SPH.

We choose PUFF \citep{searcy1998puff} as the VATD model among several available ones \citep[e.g.][]{searcy1998puff,schwaiger2012ash3d} because the ``restart feature" of PUFF makes it technically easier to incorporate exising plume.
%1D plume model bent \citep[e.g.][]{bursik2001effect} and a 
It is necessary to mention that the initial condition adopted in PUFF is a 3D ash cloud other than line source which is more commonly used in other VATD models.
A 3D plume model, Plume-SPH \citep{gmd-2017-119}, is adopted to create initial condition for PUFF. As a first-principle-based 3D numerical model, Plume-SPH not only provides more accurate prediction and new explanations for many features of explosive volcanism but also naturally create a 3D ash cloud, which can   serves as initial condition for VATDs. However, to the best of our knowledge, there are no VATD simulations taking such 3D ash clouds as initial condition yet. Our focus in this chapter is to explore conducting VATDs simulation using initial ash cloud based on 3D plume simulation.

\subsection{The Pinatubo eruption}

The 1991 eruption of Mountain Pinatubo is used as case study in this chapter. The Mountain Pinatubo erupted between June 12 and 16, 1991, after weeks of precursory activity. The climactic phase starts at June 15 0441 UTC and ends around 1341 UTC \citep{holasek1996satellite}. This climactic phase generated voluminous pyroclastic flows and sent Plinian and co-ignimbrite ash and gas columns to great altitudes \citep{scott1996pyroclastic}. The evolution of the Pinatubo ash and $SO_2$ clouds was tracked using the visual spectrum \citep{holasek1996satellite}, the ultraviolet spectrum with a total ozone mapping spectrometer (TOMS) \citep{guo2004re}, and the infrared spectrum with a variety of sensors, including advanced very high-resolution radiometer (AVHRR) \citep{guo2004particles}. There are also sufficient observational data to estimate the eruption condition for climactic phase of Pinatubo eruption \citep{suzuki2009three}. The availability of calibrated eruption condition and extensive observational data makes the Pinatubo eruption an ideal case for our research. In addition, this case is also used in chapter \ref{chapter:case-studies} for model validation. Data for the plume is readily available.

\section{Setting up simulations} \label{sec:Methodology}

One plume model and one VATDs is used in this study.
%The first plume model is a 1D volcanic eruption column model, bent \citep{bursik2001effect}. In producing its eruption outputs, bent accounts for atmospheric (wind, temperature, humidity, etc.) conditions as given by atmospheric sounding data. Thus plume rise height is given as a function of volcanic source and environmental conditions. 
The plume model is Plume-SPH \citep{gmd-2017-119}, the software developed in this thesis. Its output can be directly taken by VATDs that based on Lagrangian method as the initial ash cloud. A VATDs, PUFF \citep{tanaka1991development,searcy1998puff}, is used to propagate ash parcels in the wind field taking the initial ash clouds as input. In the case study, we use global NOAA/OAR/ESRL 6-h, 2.0° reanalysis wind fields data \citep{whitaker2004reanalysis, compo2006feasibility, compo2011twentieth}.

\subsection{Creating of initial ash cloud}

Key global descriptors of plume, are traditionally used to define the initial condition for VATD models. Key global descriptors of plume include maximum and minimum plume height, vertical spread (a vertical scale over which the ash will be distributed), column radius. There are two different ways to determine these parameters for VATDs. The first method determines key global descriptors of plume directly by calibrating these parameters so that the simulated ash evolution match as much as possible with observation of ash transportation \citep[e.g.][]{fero2008simulation,fero2009simulating}. However, the calibrated parameters might departure from observation. For example, the calibrated maximum height of Pinatubo eruption ($<25 km$) is reported to be much lower \citep{fero2009simulating} than observed top height of around $40 km$. It is also a common practice that VATDs take output of 1D plume models as key global descriptors of plume. For example, \citet{bursik2012estimation} and \citet{ stefanescu2014temporal} use bent \citep{bursik2001effect} to generate global descriptors. The new method proposed in this chapter generates initial ash cloud directly from output of 3D plume model, which has never been attempted.
%
%\begin{figure}
%\center
%    \begin{minipage}{.325 \textwidth}
%        \centering
%        \includegraphics[width=0.99 \textwidth]{Chapter-7/Figures/cut_view_ptype}
%    \end{minipage}%
%    \begin{minipage}{.325 \textwidth}
%        \centering
%        \includegraphics[width=0.99 \textwidth]{Chapter-7/Figures/cut_view_ptype_filtered}
%    \end{minipage}%   
%    \caption{The first step to creat initial ash cloud from output of Plume-SPH. Colour indicates particle types, the red particles are phase 2, blue particle are phase 1. Picture to the left is visualization of "raw data". Pictures to the right is visualization of filtered data. Only particles of phase 2 are left.}
%    \label{fig:Plume-SPH-filter-phase}
%\end{figure}
\begin{figure}
\center
\includegraphics[width=0.45\textwidth]{./PPT/Creat_initial_Ash}
\caption{The steps to create initial condition for PUFF based on raw output of Plume-SPH.}
\label{fig:create-initial-ash-plume-sph}
\end{figure}
The steps of creating initial ash cloud are shown in Fig. \ref{fig:create-initial-ash-plume-sph}.
Output of Plume-SPH include two types of SPH particles, particle of phase 1 to represent ambient air and particles of phase 2 to represent erupted material. We use SPH particle of phase 2 to create initial ash cloud. After reaching to the maximum height and starting spreading horizontally, particles of phase 2 forms an initial ash cloud. See, Fig. \ref{fig:Plume-SPH-Pinatubo-ash-cloud}, the initial ash cloud of the Pinatubo eruption at around $600s$ after eruption. 3D plume simulation is considered being completed at this point. The lower part of the plume keeps moving vertically upwards and hence is involved only minimally in horizontal ash transport. Thereby, we define only the upper part of the plume as initial ash cloud based on an elevation threshold. Considering SPH particles are essentially discretization points and have no size, we need to assign grain size to each particle according to given distribution, mean and standard deviation of grain size \citep{paladio1996tephra}. As a consequence, the discretization points (has no size) are switch to Lagrangian tracers with sizes. The coordinates of these tracers, which are in the local Cartesian coordinate system of Plume-SPH, need to convert into coordinates in PUFF's global coordinate system, which is writen in terms of $(longitude, latitude, height)$. PUFF then takes the initial ash clouds consisting of batch of Lagrangian tracers and propagates from time $t$ to time $t+\Delta t$ via an advection/diffusion equation \citep{searcy1998puff}.
\begin{equation}
\textbf{R}_i(t+\Delta t) = \textbf{R}_i(t) + \textbf{W}(t)\Delta t + \textbf{Z}(t)\Delta t +  \textbf{S}_i(t) \Delta t
\end{equation}
Here, $\textbf{R}_i(t)$ is the position vector of $ith$ Lagrangian tracer at time $t$, $\textbf{W}$ accounts for wind advection, $\textbf{Z}$ accounts for turbulent dispersion and $\textbf{S}$ is the terminal gravitional fallout velocity, which depends on tracer's size.
To summarize, it takes four steps to create initial ash cloud from output of Plume-SPH:
\begin{itemize}
\item filtering by SPH particle type to select SPH particles that represent erupted material
\item filtering by an elevation threshold to select these particles involved in horizontal spreading 
\item switching discretization points to Lagrangian tracers by assigning grain size to each particle
\item converting coordinates of Lagrangian tracers into VATDs' coordinate system
\end{itemize}
The volcanic plume and initial ash clouds used in the case study are shown in Fig. \ref{fig:Plume-SPH-Pinatubo-ash-cloud}. We have to point out that since both Plume-SPH and PUFF are based on Lagrangian method, no extra step is required to convert results on Lagrangian particles onto Eulerian grid.

\begin{figure}[!htb]
    \centering
    \begin{minipage}{.325\textwidth}
        \centering
        \includegraphics[width=0.99 \textwidth]{Chapter-7/Figures/mssfrc_front-filter-by-phase}
    \end{minipage}%
    \begin{minipage}{.325 \textwidth}
        \centering
        \includegraphics[width=0.99 \textwidth]{Chapter-7/Figures/mssfrc_top-with-axis}
    \end{minipage}%
    \begin{minipage}{.325 \textwidth}
        \centering
        \includegraphics[width=0.99 \textwidth]{Chapter-7/Figures/mssfrc_front-z15000}
    \end{minipage}%   
    \caption{Red represents high mass fraction of erupted material while blue represents low mass fraction. All particles in the pictures are of type phase 2 (phase 1 has been removed in step 1) at 600s after eruption, at which time, the plume has already reached the maximum height and started spreading radially. The radially expanding portion of the plume is extracted (step 2) as initial ash cloud. Pictures from left to right are: front view of the whole plume, top view of the plume and front view of the initial ash cloud, which is essentially portion of the whole plume with elevation higher than a given threshold (in this picture is $15000 m$)}
    \label{fig:Plume-SPH-Pinatubo-ash-cloud}
\end{figure}

Table \ref{tab:VATDs-source-term-determination} compares three different methods for creating initial conditions for VATDs: 1) creating initial condition based on parameter calibration without any plume model (method 1), 2) creating initial condition based on output of 1D plume model (method 2), 3) extracting initial ash cloud from 3D plume simulation (method 3). The first method determines all parameters by calibration. Then create initial line source or ash cloud according to semiempirical plume shape expression, such as Poisson or Suzuki. Both other two methods depend on plume models. Compared with 3D plume models, even though 1D plume models also account for basic conservation laws, they use entrainment coefficients to account for entrainment of air due to turbulent mixing and crosswind field. These coefficients are usually calibrated according to well studied cases. The feedback from plume to atmosphere is ignored. What's more, 3D plume model can generate initial ash cloud in a 3D space while 1D plume models still need semiempirical expression for vertical ash distribution to create 3D initial ash cloud. In addition, the number of Lagrangian tracers is a free parameter when using semiempirical plume shape expressions while it depends on simulation when extracting initial condition from 3D plume simulation results.

\begin{table}
\centering
      \caption{Three different methods for creating initial conditions (initial ash clouds) for PUFF simulation}		
	  \begin{tabular}{p{33mm}p{32mm}p{32mm}p{32mm}}
	    \hline
	    		 & No model & 1D model & 3D model \\
	    		 \hline    		 
	  Maximum height & Observation $\&$ \newline Calibration & Semiempirical &  1st principle \\
	  Average height &  Calibration & Conservation \newline laws (1D) &  1st principle  \\
	  Vertical spread &  Calibration & Semiempirical & 1st principle \\
	  Column radius & Calibration  &  Conservation \newline laws (1D) &  1st principle \\
	  Plume shape & Semiempirical & Semiempirical  & 1st principle \\
	  Tracers number & Free \newline parameter  & Free \newline Parameter & Based on \newline simulation\\ 
	    \hline
	  \end{tabular}
	  \label{tab:VATDs-source-term-determination}
\end{table}

\subsection{Restart PUFF}

The plume development and ash transportation are of different time scale and length scale. Resolution of plume simulating is much finer than that of ash transportation.
It takes around $600 s$ for Pinatubo plume to reach a steady height. However the duration of climactic phase of eruption persists for a few hours (9 hours in this case). It is computationally too expensive to do 3D plume simulation up to several hours. In order to handle the difference in time scale, we mimic successive eruption with intermittent pulsing releasing of ash particles. Particularly, we restart PUFF at an interval of $600 s$, which is physical time of plume simulation. At every restart, we integrate the output of last PUFF simulation and the ash cloud obtained from output of Plume-SPH into a new ash cloud. This new ash cloud serves as new initial condition when restart PUFF. The interval of the pulsing release equals to the simulation time of plume model, that is $600 s$ in our case study. The total number of Lagrangian tracers used in PUFF equals to summation of numbers of particles in all releases. So the total number of tracers is not a parameter selected by user any more.
\citet{fero2008simulation} proposed using more realistic time-dependent plume heights. We do not adopt that strategy here due to lack of well-estimiated time-dependent eruption conditions, although the idea is straightforward to consider.

\begin{figure}
\center
\includegraphics[width=0.90 \textwidth]{Chapter-7/Figures/Restart-PUFF.pdf} 
    \caption{Mimic successive eruption with intermittent pulsing releasing of ash particles. $t_I$ is the period of pulsing release. $t_I$ equals to physical time of 3D plume simulation.}
    \label{fig:Restart-Puff}
\end{figure}

\subsection{Sensitivity analysis of other parameters}

Besides key global descriptors of plume, other input parameters for PUFF simulation are: horizontal diffusivity, vertical diffusivity, mean grain size, grain size standard deviation, particles' number and eruption duration. We present in this subsection sensitivity studies on these parameters. We also investigate the influence of eruption duration.

A systematic sensitivity study is applied to check how other input parameters would influence the simulation results. Such sensitivity analyses will serve as basis for identifying possible sources of disparities between simulation and observation.
The sensitivity analyses illustrate that adjustment of other input parameters produces negligible visual differences in ash cloud distribution. Using different vertical diffusivities in range of $[100, 100000] m^2s^{-1} $ and different horizontal diffusivities in range of $[1, 20] m^2s^{-1}$ produces visually negligible differences. 
The eruption duration should depend on the actual eruption duration (or the duration of climactic phase) of a specific eruption. We conducted several simulations with eruption duration varying in range of $[5, 11] hours$ with slight different starting time of climactic phase. Table \ref{tab:Pinatubo-eruption-duration} lists all these simulations. However, only tiny visible differences are observed among the simulated ash transportation. The mean of grain size also has visually ignorable effects on long-term ash transportation according to our sensitivity tests varying the log mean (base 10) grain radius in a range of $[-7.3, -3.5] m$. 
The standard deviation, when varying in range of $[0.1, 10]$, generate ignorable difference on long-term ash transportation as well. Similar conclusion on parameter sensitivity is reported by \citet{fero2008simulation}.
Among these parameters, the eruption duration and beginning time shows, even though tiny, the most obvious influence on simulated ash distribution. In order to show such differences in an intuitive way, Fig. \ref{tab:Pinatubo-eruption-duration} shows simulated ash distribution corresponding to 4.9 hours duration, 9 hours duration and 11 hours duration respectively. After 72 hours, relative to the simulation starting time, these three cases generate generally similar results, with high concentration ash covers almost the same region. The difference of lower concentration distribution is relatively more obvious. Ash cloud covers broadest area when eruption duration is 11.1 hours. To summarize, all parameters other than key global descriptors have either tiny or ignorable affects on long-term ash distribution simulation.

\begin{table}[htp]
\centering
      \caption{The starting and ending time (UT) for simulating the climactic phase of Pinatubo eruption on June 15 1991. Observed plume height \citep{holasek1996satellite} at different time are also list in the table.}		
	  \begin{tabular}{p{45mm}p{23mm}p{23mm}p{23mm}p{23mm}}
	    \hline
        Eruption duration & 4.9 hours & 9 hours & 10 hours & 11.1 hours \\
	    \hline
	    Start time & 0441 & 0441 & 0441 & 0334 \\
	    Height at start time & 37.5 km & 37.5 km  & 37.5 km  & 24.5 km \\
	    
	    End time   & 0934 & 1341 & 1441 & 1441  \\
	    	Hheight at end time & 35 km & 26.5 km & 22.5 & 22.5 km \\
	    \hline
	  \end{tabular}
	  \label{tab:Pinatubo-eruption-duration}
\end{table}

\begin{figure}[!htb]
    \centering
    \begin{minipage}{.325\textwidth}
        \centering
        \includegraphics[width=0.99 \textwidth]{Chapter-7/Figures/199106180441-ash-5hr-cut15000}
    \end{minipage}%
    \begin{minipage}{.325 \textwidth}
        \centering
        \includegraphics[width=0.99 \textwidth]{Chapter-7/Figures/199106180441-ash-9hr-cut15000}
    \end{minipage}%
    \begin{minipage}{.325 \textwidth}
        \centering
        \includegraphics[width=0.99 \textwidth]{Chapter-7/Figures/199106180441-ash-11hr-cut15000}
    \end{minipage}%   
    \caption{Simulated ash cloud distribution corresponding to eruption duration of 4.9 hours, 9 hours and 11.1 hours (from left to right) respectively. Starting and ending time for each case is in Table \ref{tab:VATDs-source-term-determination}. The contours are for ash distribution at 72 hours after eruption.}
    \label{fig:PUFF-sensitivity-duration}
\end{figure}

The new methodology of generating initial ash cloud introduces another new parameter: elevation threshold. We also carry out sensitivity analysis on this parameter by varying the elevation threshold from $1500 m$ (the height of the vent) to $25000 m$. The simulated ash distributions show obviously visible differences. Such influence is especially obvious when the elevation threshold is either very large or very small. However, varying the elevation threshold in the range of $[12000, 18000] m$ generates relatively small difference in ash distribution. Figure \ref{fig:PUFF-sensitivity-elevation-threshold} compares the simulated ash distribution corresponding to elevation thresholds of $1500 m$ and $15000 m$. Compared with ash distribution for threshold of $15000 m$, an extra long tail appears when using elevation threshold of $1500 m$. Adopting smaller elevation thresholds essentially adds more tracers at lower elevation. As the wind at different elevation are different, these newly added tracers at lower elevation would transpose to different directions.

\begin{figure}[!htb]
    \centering
    \begin{minipage}{.325\textwidth}
        \centering
        \includegraphics[width=0.99 \textwidth]{Chapter-7/Figures/199106180441-ash-9hr-cut1500}
    \end{minipage}%
    \begin{minipage}{.325 \textwidth}
        \centering
        \includegraphics[width=0.99 \textwidth]{Chapter-7/Figures/199106180441-ash-9hr-cut15000}
    \end{minipage}%  
    \caption{Simulated ash distribution taking initial ash clouds obtained using different elevation thresholds ($1500 m$ and $15000$ m) from output of Plume-SPH. The contours are corresponding to ash concentration at 72 hours after eruption. The starting and ending time are corresponding to 9 hours duration case in Table \ref{tab:Pinatubo-eruption-duration}}
    \label{fig:PUFF-sensitivity-elevation-threshold}
\end{figure}

The sensitivity analyses demonstrate that the initial condition for VATDs has the most significant effect on simulated ash distribution while all other input parameters have either tiny or ignorable influence. The initial ash cloud generated based on semiempirical expression, which is a function of several parameters, might be significantly disparate from realistic ash cloud. Such initial condition might greatly compromise the accuracy of VATDs simulation.

\section{Results and discussion}

Ash transportation are simulated based on either initial condition generated according to key global descriptors of plume and semiempirical plume shape expression or initial ash cloud extracted from 3D plume simulation (Plume-SPH). The results are compared against observational ash clouds to evaluate the accuracy of each simulation. 
%To identify the source of disparity between simulation and observation of long-term ash transportation, PUFF simulation results are also compared against observed $SO_2$ clouds.
The plume development simulation is described in detail in Chapter 6. 
%A 3D initial ash cloud is created based on the outcome of plume simulation using Plume-SPH.
The steps for creating initial 3D ash cloud and for using the 3D ash cloud in VATD models have been described in detail in Section \ref{sec:Methodology}.
As for another way to create initial condition, the observed maximum height along with several other parameters assigned  semiempirically \citep{bursik2012estimation} are plugged into an Poisson distribution expression to vertically distribute ash particles.
%1D plume simulation (using bent) adopts exactly the same eruption parameters, material properties and atmosphere as adopted by Plume-SPH (Table \ref{tab:material_properties}, Table \ref{tab:input_parameters}).
%For ash transportation simulation, initial condition is defined based on plume simulation output. Two parameters obtained directly from bent simulation, the horizontal spread and height of plume, are used to define the initial ash cloud together with several other parameters assigned semiempirically \citep{bursik2012estimation}. 
See details in Table \ref{tab:input_parameter_PUFF_simulation}. The output of Plume-SPH is a group of SPH particles in 3D space, from which the initial ash cloud can be directly extracted based on an elevation threshold. The simulation parameters that control the simulation characteristics are the same for both simulations. As has been shown in sensitivity analyses section, these parameters have less influence on simulation results compared with initial condition. Details can be found in Table \ref{tab:input_parameter_PUFF_simulation}.

\begin{table}[htp]
\centering
      \caption{Parameters for PUFF simulation of the climactic phase of Pinatubo eruption on June 15 1991. The first six parameters define the initial condition. First six parameters for Plume-SPH are based on plume modeling but not needed for creating initial ash cloud. }
	  \begin{tabular}{lrrr}
	    \hline
	    Parameters    & Unit & Semiempirical & Plume-SPH \\
	    \hline
	    Maximum Height ($H_{max}$) & $m$ & 40000 & 41800 \\
	    %Minimum Height & $m$ & 28019.7 & - \\
	    Horizontal Spread & $km$ & 103.808 & -\\
	    Vertical Spread ($H_{width}$) & $km$ & 6.662  & 20 \\
	    Plume Shape & - & Poisson & - \\
	    Total Ash Particles  & - & 1768500 & 1768500 \\
	    Elevation Threshold & $m$ & - &  15000 \\
	    Horizontal Diffusivity & $m^2/s$ &10000 & 10000\\
	    Vertical Diffusivity & $m^2/s$ & 10 & 10 \\
	    Grain Size Distribution & - & Gaussian & Gaussian  \\
	    Mean of Grain Size (Radius) & $mm$ & $3.5 \times 10 ^-2$ & $3.5 \times 10 ^-2$ \\
	    Standard Deviation of Grain Size & - &  1.0 & 1.0 \\
	    	Start Time & UT & 0441 & 0441 \\
	    End time & UT & 1341 & 1341 \\
	    Simulation Duration & hour & 120 & 120 \\
	    \hline
	  \end{tabular}
	  \label{tab:input_parameter_PUFF_simulation}
\end{table}

\subsection{``Plume-SPH+PUFF" and ``Semiempirical initial cloud +PUFF"}

Simulation results based on input parameters list in Table \ref{tab:input_parameters}, Table \ref{tab:material_properties} and Table \ref{tab:input_parameter_PUFF_simulation} are compared with TOMS images and AVHRR BTD ash cloud map in Fig. \ref{fig:Plume-SPH-PUFF-ash-cloud}.

\begin{figure}[!htb]
    \centering
    \begin{minipage}{.325\textwidth}
        \centering
        \includegraphics[width=0.99 \textwidth]{Chapter-7/Figures/bent-23hr-ash}
    \end{minipage}%
    \begin{minipage}{.325 \textwidth}
        \centering
        \includegraphics[width=0.99 \textwidth]{Chapter-7/Figures/SPH-Plume-23hr-ash}
    \end{minipage}%
    \begin{minipage}{.325 \textwidth}
        \centering
        \includegraphics[width=0.99 \textwidth]{Chapter-7/Figures/OB-ash-23hr-ash}
    \end{minipage}% 
    \\
        \begin{minipage}{.325\textwidth}
        \centering
        \includegraphics[width=0.99 \textwidth]{Chapter-7/Figures/bent-31hr-ash}
    \end{minipage}%
    \begin{minipage}{.325 \textwidth}
        \centering
        \includegraphics[width=0.99 \textwidth]{Chapter-7/Figures/SPH-Plume-31hr-ash}
    \end{minipage}%
    \begin{minipage}{.325 \textwidth}
        \centering
        \includegraphics[width=0.99 \textwidth]{Chapter-7/Figures/OB-ash-31hr-ash}
    \end{minipage}% 
    \\
        \begin{minipage}{.325\textwidth}
        \centering
        \includegraphics[width=0.99 \textwidth]{Chapter-7/Figures/bent-55hr-ash}
    \end{minipage}%
    \begin{minipage}{.325 \textwidth}
        \centering
        \includegraphics[width=0.99 \textwidth]{Chapter-7/Figures/SPH-Plume-55hr-ash}
    \end{minipage}%
    \begin{minipage}{.325 \textwidth}
        \centering
        \includegraphics[width=0.99 \textwidth]{Chapter-7/Figures/OB-ash-55hr-ash}
    \end{minipage}% 
    \caption{Comparison between ``Semiempirical initial cloud +PUFF" and ``Plume-SPH+PUFF". Pictures from left to right are: PUFF simulation based on initial condition created according semiempirical plume shape expression, PUFF simulation based on initial condition generated by Plume-SPH, TOMS or AVHRR image of Pinatubo ash cloud. Ash clouds at different hours after eruption are on different rows. From top to bottom, the images are corresponding to around 23 hours after eruption (UT 199106160341), 31 hours after eruption (UT 199106161141), 55 hours after eruption (UT 199106171141). The observation data on the first row are TOMS ash and ice map. The observation data on the second and third row are AVHRR BTD ash cloud map with atmospheric correction method applied \citep{guo2004particles}.}
    \label{fig:Plume-SPH-PUFF-ash-cloud}
\end{figure}

The difference in simulation results between ``Semiempirical initial cloud +PUFF" and ``Plume-SPH+PUFF" is obvious. The simulated ash concentration based on initial condition created from Plume-SPH is much closer to observation than that based on semiempirical plume shape expression. Around 23 hours and 31 hours after the beginning of the climactic phase, ``Plume-SPH+PUFF" simulation generates ash images that generally close to observational image, especially the location where the high concentration ash presents. However, these ash at near west to Pinatubo mountain observed in satellite images does not present in ``Plume-SPH+PUFF" simulation results. This disparity is very possible due to the fact that the Mountain Pinatubo continued erupting after climactic phase while our simulation only simulates the climactic phase. The ash released after climactic phase is not accounted in the simulation results. The ``Semiempirical initial cloud +PUFF" simulation, however, forecasts an ash distribution faster and narrower than observation. The location, where the high concentration ash presents, locates to the northwest of observed ash. 
Around 55 hours after the beginning of the climactic phase, the disparity between observation and simulation becomes more obvious. Ash distribution of ``Semiempirical initial cloud +PUFF" simulation locates far west to the observed ash. The high concentration area of ``Plume-SPH+PUFF" simulation, even though closer to observation than that of ``Semiempirical initial cloud +PUFF", is still more west than observation.

%Both ``bent+PUFF" simulation and ``Plume-SPH+PUFF" simulation are conducted using the same eruption condition, material properties and atmosphere condition for plume simulation. 
At the stage of ash transportation simulation, except for the initial condition, both simulations adopt the same parameters. The number of ash particles and particle size distribution are the same for both. That is to say, the only difference between these two simulations is the initial condition. Recall the conclusions we drew in the sensitivity analyses section that the initial condition has most significant influence on ash transportation simulation. It is clear that the big difference between simulation results of ``Plume-SPH+PUFF" and ``Semiempirical initial cloud +PUFF" is attribute to the initial condition.
%It is nature to suspect that the poorer prediction of the Pinatubo plume development by bent leads to poorer initial condition and hence poorer PUFF simulation. However, this is not true. 
%One of the key global property of plume, the maximum height, is choosen to be consistent with observation \citep{lynch1996mount}. Local variables, such as radius, temperature, pressure, mass fraction of entrained air, as functions of elevation are compared in a recent internal plume models comparison study \citep{costa2016results}. Bent shows generally comparable results with other 1D and 3D plume models. It has been illustrated that 1D plume models including bent, with proper entrainment coefficients calibration, are capable of predicting global key features, such as maximum height, and local variable of volcano plumes. 
The maximum height simulated by Plume-SPH, see Fig. \ref{fig:Plume-SPH-Pinatubo-ash-cloud} is around $41800 m$, which is also very close to observation. So out suspecting is that the semiempirical way of creating initial conditions induced the large disparity between ``Semiempirical initial cloud +PUFF" simulation results and observation.

\subsection{Calibration of maximum height}

It might be the semiempirical way of creating initial ash cloud based on assumed plume shape that contributes to the large differences between ``Semiempirical initial cloud +PUFF" and observation. In this section, we majorly focus on the vertical distribution of ash particles in initial ash cloud. The vertical ash distribution created according to assumed plume shape based on observed maximum height might be significantly away from real ash distribution.
The majority of volcanic ash particles usually present a lower elevation than maximum height. For instance, \citet{holasek1996satellite, holasek1996experiments} reported the maximum Pinatubo plume height as high as around $39 km$ while the cloud heights were estimated at $20 \sim 25 km $, \citet{self1993atmospheric} report the maximum plume height could be $>35 km$ and the plume heights are $23 \sim 28 km$ after $15 \sim 16$ hours. The neutral buoyant regions of the Pinatubo aerosol estimated by different measurements are: $17 \sim 26 km$ (lidar) by \citet{defoor1992early}, $20 \sim 23 km$ (balloon) by \citet{deshler1992balloonborne}, $17 \sim 28 km$ (lidar) by \citet{jager1992pinatubo}, and $17 \sim 25 km$ (lidar) by \citet{avdyushin19931}. Based on comparison between simulated cloud with early infrared satellite images of Pinatubo, \citet{fero2008simulation} reported that the majority of ash was transported between $16$ and $18 km$. This is physically understandble as particles are concentrated along intrusion height of umbrella cloud, not near top because the plume top is just momentum overshoot.

Initial ash cloud created based on semiempirical plume shapes usually tends to distribute ash particles at higher elevation even when adopting a maximum height that is close to real maximum height. Here we check two commonly used plume shapes, the Poisson and Suzuki.
For Poisson plume shape, the vertical height of ash particles are determined according to Eq. (\ref{eq:Poisson-plume-shape}).
\begin{equation}
H=H_{max} - 0.5 H_{width}*P+H_{width}R
\label{eq:Poisson-plume-shape}
\end{equation}
where $P$ is an integral value drawn from a Poisson
distribution of unit mean, $R$ is a uniformly distributed random number between 0 and 1, $H_{max}$ is the maximum plume height, $H_{width}$ represents an approximate vertical range over which the ash will be distributed.
For Suzuki plume shape \citep{suzuki1983theoretical}, volcano ash mass is assumed to distribute vertically following the Suzuki equation (Eq. (\ref{eq:Suzuki-plume-shape})).
\begin{equation}
Q(z)=Q_m* \frac{k^2(1-z/H_{max})exp\left(k(z/H_{max} -1 )\right)}{H_{max}\left[1-(1+k)exp(-k)\right]}
\label{eq:Suzuki-plume-shape}
\end{equation}
Where $Q_m$ is the total mass of erupted material, $k$ is shape factor, which is an adjustable constant that controls ash distribution with height. A low value of $k$ gives a roughly uniform distribution of mass with elevation, while high values of $k$ concentrate mass near the plume top.
Particle distribution (in terms of mass percentage or particle number percentage) in vertical direction in the initial ash cloud are shown in Fig. \ref{fig:Particle-distribution-Plume-SPH-vs-semiempirical}. In that figure, the particle distribution based on Plume-SPH output are compared with particle distribution created based on semiepirical shape expressions. The key descriptors of the plume predicted by bent are used in semiepirical shape expressions. When adopting Poisson plume shape, the majority of the particles are between $30 km \sim 40 km$. Obviously, Poisson distributes majority ash at a much higher elevation than observation. As for Suzuki, the majority of ash particles also distribute in a range that significantly higher than $25 km$. As for initial ash cloud based on Plume-SPH simulation, the major population of ash particles distribute between $17 km \sim 28 km$, which match well with observations. The maximum height is also consistent with observation. To summarize, the unrealistic initial ash cloud might be generated by semiempirical plume shape expression even when using realistic plume maximum height.

\begin{figure}[!htb]
    \centering
    \begin{minipage}{.247\textwidth}
        \centering
        \includegraphics[width=0.99 \textwidth]{Chapter-7/Figures/Plume-SPH-ParticleDis-NoTrucation-z}
    \end{minipage}%
    \begin{minipage}{.247 \textwidth}
        \centering
        \includegraphics[width=0.99 \textwidth]{Chapter-7/Figures/Plume-SPH-ParticleDis-z}
    \end{minipage}%
    \begin{minipage}{.247 \textwidth}
        \centering
        \includegraphics[width=0.99 \textwidth]{Chapter-7/Figures/Possion-Hmax40k-ParticleDis-z}
    \end{minipage}% 
    \begin{minipage}{.247 \textwidth}
        \centering
        \includegraphics[width=0.99 \textwidth]{Chapter-7/Figures/Suzuki-Hmax40k-ParticleDis-z}
    \end{minipage}% 
    \caption{Particle distribution of initial ash cloud in vertical direction. The picture to the left is corresponding to initial ash cloud obtained from Plume-SPH output. The second picture is corresponding to ash distribution truncated by a elevation threshold of $15000 m$. Third picture is corresponding to Poisson distribution with maximum height based on bent simulation. Another parameter, the vertical spread, in the expression of Poisson plume shape is $6662 m$. The picture to the right is corresponding to Suzuki distribution with maximum height based on bent simulation. Another parameter in Suzuki distribution, the shape factor, is $4$. The $x$ axis is the percentage of particle number for Plume-SPH and Poisson, is the percentage of erupted material mass percentage for Suzuki.}
    \label{fig:Particle-distribution-Plume-SPH-vs-semiempirical}
\end{figure}

For Poisson and Suzuki plume shape, vertical distribution of ash particles can't be lower down without changing the maximum height. To distribute majority population of ash particles at lower elevation, the maximum height has to be reduced to a value smaller than observed maximum height. This strategy is the same as the traditional source term calibration method. A set of initial ash clouds using different maximum heights based on Poisson plume shape assumption is shown in Fig. \ref{fig:Particle-distribution-Plume-calibrate-semiempirical}). The maximum heights, by no means, are obtained from any plume model or observation. Except for maximum height, all other parameters for creating initial ash cloud are the same as these in Table \ref{tab:input_parameter_PUFF_simulation}. The range between which major populations of ash particles locate is different when using different maximum heights. These ash clouds based on different maximum heights are then used as initial condition in PUFF simulation, whose results are show in Fig. \ref{fig:Various-Maximum-height-Pinatubo-ash-cloud}.

\begin{figure}[!htb]
    \centering
    \begin{minipage}{.247 \textwidth}
        \centering
        \includegraphics[width=0.99 \textwidth]{Chapter-7/Figures/Possion-Hmax10k-ParticleDis-z}
    \end{minipage}%
    \begin{minipage}{.247 \textwidth}
        \centering
        \includegraphics[width=0.99 \textwidth]{Chapter-7/Figures/Possion-Hmax20k-ParticleDis-z}
    \end{minipage}% 
    \begin{minipage}{.247 \textwidth}
        \centering
        \includegraphics[width=0.99 \textwidth]{Chapter-7/Figures/Possion-Hmax30k-ParticleDis-z}
    \end{minipage}% 
    \begin{minipage}{.247 \textwidth}
        \centering
        \includegraphics[width=0.99 \textwidth]{Chapter-7/Figures/Possion-Hmax35k-ParticleDis-z}
    \end{minipage}% 
    \caption{Particle distribution based on Poisson plume shape in vertical direction with different maximum heights. Pictures from left to right are corresponding to maximum height of $10000 m$,  $20000 m$,  $30000 m$, $35000 m$. Another parameter, the vertical spread, in the expression of Poisson plume shape is $6662 m$ for all cases. The $x$ axis is the percentage of particle number.}
    \label{fig:Particle-distribution-Plume-calibrate-semiempirical}
\end{figure}

\begin{figure}[!htb]
    \centering
    \begin{minipage}{.245\textwidth}
        \centering
        \includegraphics[width=0.99 \textwidth]{Chapter-7/Figures/bent-55hr-ash-MaxH10000}
    \end{minipage}%
    \begin{minipage}{.245 \textwidth}
        \centering
        \includegraphics[width=0.99 \textwidth]{Chapter-7/Figures/bent-55hr-ash-MaxH20000}
    \end{minipage}%
    \begin{minipage}{.245 \textwidth}
        \centering
        \includegraphics[width=0.99 \textwidth]{Chapter-7/Figures/bent-55hr-ash-MaxH30000}
    \end{minipage}%  
    \begin{minipage}{.245 \textwidth}
        \centering
        \includegraphics[width=0.99 \textwidth]{Chapter-7/Figures/bent-55hr-ash-MaxH35000}
    \end{minipage}%   
    \caption{Ash transportation simulated by PUFF using different initial ash cloud created with different maximum heights. Pictures from left to right are: PUFF simulation with maximum plume height of $10000 m$, $20000 m$, $30000 m$ and $35000 m$. The images are corresponding to around 55 hours after eruption (UT 199106171141).  See the observed cloud image in Fig. \ref{fig:Plume-SPH-PUFF-ash-cloud}.}
    \label{fig:Various-Maximum-height-Pinatubo-ash-cloud}
\end{figure}

Figure \ref{fig:Various-Maximum-height-Pinatubo-ash-cloud} shows that the maximum height has significant influence on ash transportation simulation. When the maximum height is $10000 m$ or $20000 m$, the high concentration area is lag behind observation. While the designated maximum height is $35000 m$, the high concentration area is a little bit faster and much narrower than observation. When using maximum height of $41343.9 m$, the high concentration area is faster and narrower (see Fig. \ref{fig:Plume-SPH-PUFF-ash-cloud}). The simulated high concentration area is closest to observation when assigning a maximum height of $30000 m$. The front of volcano ash, with lower concentration, locates west to high concentration area. A lower concentration tailing area also appears in the simulation results while there is no such tail in observed image. PUFF simulation result based on calibrated maximum height of $30000 m$ shows similar footprint to, even though smaller in terms of covered area than, those of ``Pume-SPH+PUFF" simulation. We conclude that, for climactic phase of Pinatubo eruption, the initial ash cloud created with maximum height around  $30000 m$ generates best match ash distribution with observation. That is to say, a maximum height lower than real maximum height of plume is required by Poisson plume shape to distribute ash particles at the same elevation as real ash distribution. Our hypothesis regarding the source of disparity between "bent+PUFF" simulation and observation is confirmed.

In this section, only maximum height is adjusted to get more realistic ash distribution. Other parameters that also control the plume shape, such as vertical spread and plume width, if calibrated properly, could generate an initial ash cloud that closer to real one. However, the degree of freedom to adjust plume shape might still be limited. The third method for creating initial conditions based on 3D plume simulation is more adaptive to various cases and obviates semiempirical expressions regarding plume shape.

%\subsection{$SO_2$ clouds}
%In terms of long-term ash transportation simulation, for example, two days after climactic phase starts, ash distribution simulated by ``Plume-SPH+PUFF" are faster and more widely spreaded than observations. Several factors could contribute to the disparity between simulation and real distribution. Such as simplification in physics model, numerical error, initial condition and boundary conditions. We suspect that the disparity between simulation and observation of long-term ash transportation are majorly attributed to the simplification in PUFF, in which no microphysics is considered. PUFF takes into account the predominant physical processes that control particle movement, such as winds, dispersion and gravitational settling, but it does not account for smaller-scale physical processes. However, the microphysics, especially the mixed aggregation of ice and ash particles has been reported to enhance fallout \citep{guo2004particles} of Pinatubo ash particles. Thereby, both ice and ash particles declined more rapidly. Without considering the effect of mixed aggregation due to presence of ice, PUFF did poor job in long-term ash transportation forecast of Pinatubo eruption.
%
%The physics model of PUFF underestimates fallout of particles in scenarios when ice presents. The presence of ice, however, has less influence on $SO_2$ transportation. That is to say, PUFF might be used to simulate long-term $SO_2$ transportation, even though it was originally developed for volcanic ash transportation. The settling accounted in PUFF might be viewed as "mass sink" of $SO_2$ due to, for example, absorption or chemical reaction. More reasonable "mass sink" needs to be developed to adapt PUFF to be a real $SO_2$ cloud transportation model. In this section, we roughly use PUFF for $SO_2$ simulation to verify  our suspect on why we observe more obvious disparity between simulation and observation of long-term ash transportation. 
%
%\begin{figure}[!htb]
%    \centering
%    \begin{minipage}{.325\textwidth}
%        \centering
%        \includegraphics[width=0.99 \textwidth]{Chapter-7/Figures/SPH-Plume-7p3hr-ash}
%    \end{minipage}%
%    \begin{minipage}{.325 \textwidth}
%        \centering
%        \includegraphics[width=0.99 \textwidth]{Chapter-7/Figures/OB-SO2-7p3hr-ash}
%    \end{minipage}% 
%    \\
%    \begin{minipage}{.325\textwidth}
%        \centering
%        \includegraphics[width=0.99 \textwidth]{Chapter-7/Figures/SPH-Plume-23hr-ash}
%    \end{minipage}%
%    \begin{minipage}{.325 \textwidth}
%        \centering
%        \includegraphics[width=0.99 \textwidth]{Chapter-7/Figures/OB-SO2-23hr-ash}
%    \end{minipage}% 
%    \\
%    \begin{minipage}{.325\textwidth}
%        \centering
%        \includegraphics[width=0.99 \textwidth]{Chapter-7/Figures/SPH-Plume-31hr-ash}
%    \end{minipage}%
%    \begin{minipage}{.325 \textwidth}
%        \centering
%        \includegraphics[width=0.99 \textwidth]{Chapter-7/Figures/OB-SO2-31hr-ash}
%    \end{minipage}% 
%\\
%    \begin{minipage}{.325\textwidth}
%        \centering
%        \includegraphics[width=0.99 \textwidth]{Chapter-7/Figures/SPH-Plume-55hr-ash}
%    \end{minipage}%
%    \begin{minipage}{.325 \textwidth}
%        \centering
%        \includegraphics[width=0.99 \textwidth]{Chapter-7/Figures/OB-SO2-55hr-ash}
%    \end{minipage}% 
%\\
%    \begin{minipage}{.325\textwidth}
%        \centering
%        \includegraphics[width=0.99 \textwidth]{Chapter-7/Figures/SPH-Plume-73hr-ash}
%    \end{minipage}%
%    \begin{minipage}{.325 \textwidth}
%        \centering
%        \includegraphics[width=0.99 \textwidth]{Chapter-7/Figures/OB-SO2-73hr-ash}
%    \end{minipage}% 
%    \caption{Comparison of volcanic clouds simulated by ``Plume-SPH+PUFF" and observed $SO_2$ cloud. Pictures to the left are PUFF simulation based on Plume-SPH, pictures on the right are TOMS or AVHRR image of Pinatubo $SO_2$ cloud. $SO_2$ clouds at different hours after eruption are on different rows. From top to bottom, the images are corresponding to around 7.3 hours after starting of climatic phase (UT 199106151201), 23 hours after starting of climatic phase (UT 199106160341), 31 hours after starting of climatic phase (UT 199106161141), 55 hours after starting of climatic phase (UT 199106171141), 73 hours after starting of climatic phase (UT 199106180541).}
%    \label{fig:Plume-SPH-Pinatubo-SO2-cloud}
%\end{figure}
%
%We observe in Fig. \ref{fig:Plume-SPH-Pinatubo-SO2-cloud} that the simulated ash distribution match well with observed $SO_2$ cloud after more than one day since the beginning of climactic phase. At around 7.3 hours and 21 hours after eruption, the observed $SO_2$ ash covers a obviously large area. This is due to the fact that our simulation is only for the climactic phase but satellite observation includes $SO_2$ clouds existing before the climactic phase. At around 31 hours after starting of climactic phase, the simulated cloud match well with observation. The location of high concentration area in simulation is also consistent with observation. At 55 hours and 73 hours after starting of climactic phase, the front of simulated volcano cloud is close to observation, while the tail of simulated volcano cloud locate west to the observed tail. The area of high concentration also locates to the west of observed high concentration area. Considering that Pinatubo continued eruption after climactic phase, it is understandable that tail of observed $SO_2$ clouds lag behind the simulated tail. To reduce such disparity, the eruption duration for PUFF should be extended to include pre-climactic phase and post-climactic phase. Of course, the initial ash cloud corresponding to pre-climactic phase and post-climactic phase should be different from that of climactic phase. Proper eruption conditions are needed.
%
%To conclude, the obvious disparity between observation and PUFF simulation results of long-term ash transportation is due to inability of PUFF in accounting for acceleration of ash settling caused by ice. More physics, such as the ice caused acceleration of ash particle settling, should be included so that PUFF is able to account for aggregation and predict fallout in a more accurate way. A source term, which represents losing of $SO_2$ during transportation, is necessary to extend PUFF to a $SO_2$ transportation model.

\section{Conclusion}

This chapter presented, for the first time, a new methodology to create initial conditions for volcanic ash transportation. The new method extracts initial ash cloud from 3D plume simulation while traditional methods create initial ash cloud based on semiemperical plume shape expression. Case study of Pinatubo eruption demonstrates that this new method can create more realistic initial ash cloud and improve accuracy of ash transportation forecast. In order to explain why we got much closer ash dispersal forecast merely by adopting alternative initial conditions, more investigations were conducted. Sensitivity analyses illustrate that initial condition has more significant effects on volcanic ash transportation forecast than most of the other parameters. Calibrating of the maximum height in semiemperical plume shape expression reveals how vertical ash particle distribution in initial ash cloud could affect ash transportation. 
%At the end, the disparity between simulation and observation of long-term ash transportation is investigated concluding that it is necessary for VATDs to account for aggregation due to ice to reduce disparity in Pinatubo ash transportation forecast.

This new method provides an alternative option for creating initial conditions for ash transportation simulations. Except for the disadvantage of high computational cost it helps overcome several shortcomings of existing methods.
As numerical models based on first principle, 3D plume models eliminate parameterization and hence user intervention associated with entrainment coefficients, thereby, improve forecast capacity of ash transportation simulation. More importantly, no assumption about plume shape is needed in this new method. The plume shape generated directly by 3D simulation is more realistic and adaptive to various scenarios.
Contrastingly, semiempirical plume shape expressions only have several parameters control the vertical ash particle distribution of line source or the shape of initial ash cloud, it might have difficulties to generate initial condition that close to reality for some eruptions.
For the case study of Pinatubo eruption, we got closer simulation results to observation when using maximum height of around $30 km$, which is much lower than observed plume height of $40 km$. It is an evidence that initial ash cloud created according to semiempirical plume shape would be discrepant from reality for some eruptions. As has been shown by sensitivity analyses, such disparity has much greater influence on ash evolution simulation than all other input parameters.

The full range of research issues raised by the numerical forecasting of volcanic clouds is many and diverse. We described in this chapter the effect of initial conditions on numerical forecasts of volcanic ash transportation simulation, especially the effects of vertical ash particles distribution in the initial ash cloud. 
In fact, the differences between assumed plume particle distribution and actual (or simulated by 3D plume) model are not only in vertical direction. How do other aspects of particle distribution in the initial ash cloud affect ash transportation is yet to explore. Wind field, another important factor in volcanic ash transportation simulation is not discussed in the present work, either. Some other aspects, such as small scale physical processes, even though plays less roles, might need to be included by VATDs to improve accuracy for particular eruption event. In addition, the eruption condition, hence the initial ash clouds are subject to change during eruption, even during climactic phase of eruption. More realistic, time-dependent, initial condition for VATDs can be created from 3D plume simulation with time-dependent eruption conditions.