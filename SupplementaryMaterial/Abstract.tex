% Place abstract below.
Numerical models are broadly used to understand dynamics of volcanic plume development and forecast volcano plume hazards. All existing 3D volcano plume models adopt mesh-based Eulerian numerical methods. 
SPH (smoothed particle hydrodynamics), a mesh free Lagrangian method has several advantages over currently used mesh based methods in modeling of multiphase free boundary flows like volcanic plumes. The goal of the present work is to develop a sustainable software (Plume-SPH) for volcanic plume simulation based on SPH.
This tool will provide more accurate eruption source terms to users of VATDs (Volcanic ash transport and dispersion models) greatly improving volcanic ash forecasts.
As the first version, a relatively simple physics model (a 3D dusty-gas dynamic model assuming well mixed eruption material, dynamic equilibrium and thermodynamic equilibrium between erupted material and air that entrained into the plume, and minimal effect of winds) is adopted targeting at capturing the salient features of a volcanic plume. 
 
Meanwhile, SPH is still a developing numerical tool. It is vital to investigate its advantages and potential limitations by implementing it in different problems. It has been a common practice to include different corrections with respect to standard SPH when modeling different problems. Several newly developed techniques in SPH are adopted and extended to address numerical challenges in simulating multiphase compressible turbulent flow. The code should thus be also of general interest to the community of researchers using and developing SPH based tools. In particular, Random Choice Method (RCM) is combined with SPH to include adaptive artificial viscosity. The $SPH-\varepsilon$ turbulence model is adopted to capture mixing at unresolved scales. Heat exchange due to turbulence is calculated by a Reynolds analogy and a corrected SPH is used to handle tensile instability and deficiency of particle distribution near the boundaries. I also developed methodology to impose velocity inlet and pressure outlet boundary conditions, both of which are scarce in traditional implementations of SPH.

The core solver of our model is parallelized with MPI (message passing interface) obtaining good weak and strong scalability using novel techniques for data management. Flexibility and efficiency of data add/delete/access are guaranteed using a SFCs (space-filling curves) and object creation time based indexing and hash table based storage scheme. An algorithm is designed to dynamically adjust the computational domain according to progress of simulation, which dramatically improves the computational efficiency of Plume-SPH. Several strategies are developed to obtain/restore good load balance.

Plume-SPH is first verified by 1D shock tube tests, then by comparing velocity and concentration distribution along the central axis and on the transverse cross against experimental results of JPUE (jet or plume that is ejected from a nozzle into a uniform environment). 
Profiles of several integrated variables are compared with those calculated by existing 3D plume models for an eruption with the same MER (mass eruption rate) estimated for the Pinatubo eruption on June 15th 1991. Our results are consistent with those obtained by existing 3D plume models. Analysis of the plume evolution process demonstrates that Plume-SPH is able to reproduce the physics of plume development. Finally, Plume-SPH is utilized to create initial ash cloud, which is used as initial condition by PUFF, a VATDs, for ash transportation and dispersion simulation. The results demonstrate good agreement with observations. Creating of initial condition based on output of Plume-SPH requires less assumption regarding the shape of the ash cloud and hence less user intervention improving prediction capability of volcano ash related hazards forecasts.

The documented open source code is publicly available and can be easily extended to incorporate other models of physics. Hence it is of interest to the large community of researchers investigating multiphase free boundary flows of volcanic or other origins.