\Appendix{Determining of imaginary interface in GSPH} \label{app:determine-imag-int} 
This appendix is to shown how GSPH determines the imaginary interface, the location $q^{GSPH}$ in Fig. \ref{fig:pick-up-state-GSPH-RSPH}). By choosing the middle point as the imaginary interface, we essentially adopt piece wise constant manner when constructing of Riemann problem. The following shows how the imaginary interface is determined if a piece wise linear manner is adopted for Riemann problem construction.

\section{Piece wise linear approximation of specific volume}
Given fundamental equations (Eq. (\ref{eq:GSPH-basic1}) and (\ref{eq:GSPH-basic2})) for GSPH, plug Eq. (\ref{eq:GSPH-basic1}) into Eq. (\ref{eq:SPH-fundamental-principle})
\begin{equation}
\begin{split}
<A\left(\textbf{x}\right)> 
& \approx \int_{\Omega} A\left(\textbf{x} \prime\right) w\left(\textbf{x}-\textbf{x}\prime, h\right) d\textbf{x}\prime \\
& = \int_{\Omega}  \sum_{b} m_{b} \frac{A\left(\textbf{x} \prime\right)}{\rho(\textbf{x})} w\left(\textbf{x}-\textbf{x}\prime, h\right) w(\textbf{x}\prime - \textbf{x}_{b}, h) d\textbf{x}\prime
\end{split}
\label{eq:GSPH-fundamental-principle}
\end{equation}
Equation (\ref{eq:GSPH-fundamental-principle}) reduces to fundamental SPH approximation when assuming $w(\textbf{x} - \textbf{x}_{b}, h) \approx \delta (\textbf{x} - \textbf{x}_{b})$. 
Algebraic manipulations \citep{inutsuka2002reformulation,iwasaki2011smoothed} lead to the following approximation of momentum and energy equations:
\begin{equation}
\begin{split}
&<\dfrac{d \textbf{v}_{a}}{dt}> =-\sum_{b} m_{b} \int \frac{p(\textbf{x})}{\rho(\textbf{x})^2} \\
&\left[w(\textbf{x} - \textbf{x}_b, h) \nabla w(\textbf{x} - \textbf{x}_a, h) - w(\textbf{x} - \textbf{x}_a, h) \nabla w(\textbf{x} - \textbf{x}_b, h)\right] d\textbf{x} 
\end{split}
\label{eq:gov-gsph-v-approx}
\end{equation}
\begin{equation}
\begin{split}
&<\dfrac{d e_{a}}{dt}> =-\sum_{b} m_{b} \int \frac{p(\textbf{x})}{\rho(\textbf{x})^2} (\textbf{v} - \dot{\textbf{x}}_{a}^{\ast})\\
&\left[ w(\textbf{x}- \textbf{x}_b, h) \nabla w(\textbf{x} - \textbf{x}_a, h) - w(\textbf{x} - \textbf{x}_a, h) \nabla w(\textbf{x} - \textbf{x}_b, h) \right] d\textbf{x} 
\end{split}
\label{eq:gov-gsph-e--approx}
\end{equation}
where $\dot{\textbf{x}}_{a}^{\ast}$ is time centered velocity of particle $a$, see Eq. (\ref{eq:gsph-time-centered-velocity}):
%\begin{equation}
%\dot{\textbf{x}}_{a}^{\ast} = <\textbf{v}_{a}> + \frac{\Delta t}{2} \ddot{\textbf{x}}_{a}
%\end{equation}

Evaluating of the integral in Eq. (\ref{eq:gov-gsph-v-approx}) and Eq. (\ref{eq:gov-gsph-e--approx}) depends on interpolation of the specific volume $V(q) = \frac{1}{\rho(q)}$, where $q$ is coordinate in a local coordinate system. If the specific volume is interpolated linearly in the local coordinate system, the integrals finally result in discretized formulation as showing in Eq. (\ref{eq:gov-gsph-v-with-convolution}) and Eq. (\ref{eq:gov-gsph-e-with-convolution}).
\begin{equation}
<\dfrac{d \textbf{v}_{a}}{dt}>= -\sum_{b} m_{b} p_{a b}^{\ast} \left[ V_{ab}^2(h_a) \nabla w_{a b}(\sqrt{2} h_{a}) + V_{ab}^2(h_b) \nabla w_{a b}(\sqrt{2} h_{b}) \right]
\label{eq:gov-gsph-v-with-convolution}
\end{equation}
\begin{equation}
<\dfrac{d e_{a}}{dt}>= - \sum_{b} m_{b} p_{a b}^{\ast} [\textbf{v}_{a b}^{\ast} - \dot{\textbf{x}}_{a}^{\ast}] \left[V_{ab}^2(h_a) \nabla w_{a b}(\sqrt{2} h_{a}) + V_{ab}^2(h_b) \nabla w_{a b}(\sqrt{2} h_{b}) \right]
\label{eq:gov-gsph-e-with-convolution}
\end{equation}
where, the starred quantity are interpolated value of the solution to the Riemann problem at the imaginary interface.
$V_{ab}^2(h_a)$ and $V_{ab}^2(h_b)$ are evaluated based on interpolation of specific volume in the local coordinate system, for example, linear interpolation: 
\begin{equation}
V(q) = C_{ab}q+D_{ab}
\label{eq:gsph-V-linear-interplation}
\end{equation}
then the $V^2_{ab}(h)$ is evaluated according to:
\begin{equation}
V^2_{ab}(h) = \frac{1}{4}h^2C_{ab}^2+D_{ab}^2
\label{eq:gsph-V-linear-interplation-sq}
\end{equation}
where
\begin{equation}
C_{ab} = \frac{V_a-V_b}{\Delta q_{ab}}
\label{eq:gsph-V-linear-interplation-C}
\end{equation}
\begin{equation}
D_{ab} = \frac{V_a+V_b}{2}
\label{eq:gsph-V-linear-interplation-D}
\end{equation}
In purpose of determining the position of the imaginary interface, a weighted-average $A_{ab}^{\ast}$ of generic function $A(q)$ need to be defined.
\begin{equation}
\begin{split}
& \int_{\Omega} \frac{A\left(q \right)}{\rho^2(q)} w\left(q -q_a, h\right) w(q - \textbf{x}_{b}, h) dq \\
& =A_{ab}^{\ast} \int_{\Omega} \frac{1}{\rho^2(q)} w\left(q -q_a, h\right) w(q - q_{b}, h) dq
\end{split}
\label{eq:GSPH-f-interpolation-def}
\end{equation}
Using then the linear approximation for $A(q) = q(A_a - A_b)/ \Delta q_{ab}$ and then the linear interpolation of specific volume (see Eq. (\ref{eq:gsph-V-linear-interplation})), one obtains 
\begin{equation}
A_{ab}^{\ast} = \frac{A_a - A_b}{\Delta q_{ab}} q_{ab}^{\ast} + \frac{A_a + A_b}{2}
\label{eq:GSPH-f-interpolation-evaluation}
\end{equation}
where the interface position is weitten as
\begin{equation}
q_{ab}^{\ast} = \frac{h^2 C_{ab} D_{ab}}{2V_{ab}^2(h)}
\label{eq:GSPH-f-interpolation-s-star}
\end{equation}
This $q_{ab}^{\ast}$ is used as $q_{ab}^{GSPH}$ in Fig. \ref{fig:pick-up-state-GSPH-RSPH}.