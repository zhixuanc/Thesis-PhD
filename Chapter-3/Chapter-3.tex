
\chapter{Classical SPH Method} \label{chapter:classical-SPH}

\section{Introduction}
A mesh-free Lagragian method, SPH, is selected to numerically solve the governing equations. There are several review papers by \citet{monaghan1992smoothed, monaghan2005smoothed, rosswog2009astrophysical, price2012smoothed, monaghan2012smoothed}, giving a pretty comprehensive view over SPH. We only refer here to the representation of the constitutive equations in SPH and put more focus on specific numerical techniques for volcanic plume modeling.
In the classical SPH formulation a fluid is divided into a number of particles whose location is updated at every time step. An approximation to the field variables (velocity, density and pressure) is obtained by interpolation of particle quantities using a kernel.
We will only consider kernel functions with compact support. Derivatives of variables are converted into derivatives of the kernel function, and different formulations of the derivatives can be shown to possess various properties.

\section{Fundamental principles}
There are several different ways for discretizing governing equations (PDEs or ODEs) with SPH. We present here one of them following \citet{monaghan1992smoothed, monaghan2005smoothed, monaghan2012smoothed}. The starting point of approximating a function with SPH is the translation property of the Dirac function $\delta(\textbf{x})$, for an arbitrary function $A(\textbf{x})$, the following equation holds. 
\begin{equation}
A\left(\textbf{x}\right)=\int_{-\infty}^{\infty} A\left(\textbf{x} \prime\right) \delta \left(\textbf{x} \prime - \textbf{x}\right) d\textbf{x} \prime
\label{eq:Dirac-translation}
\end{equation}
The Dirac function in Eq. (\ref{eq:Dirac-translation}) can be approximated by a weighting function $w\left(\textbf{x}-\textbf{x}\prime, h\right)$ (or $w\left(\textbf{x}\prime-\textbf{x}, h\right)$) which tends to a Dirac function when the smoothing length $h \rightarrow 0$ :
\begin{equation}
\lim _{h \rightarrow 0} w\left(\textbf{x} \prime-\textbf{x}, h\right) =  \delta \left(\textbf{x} \prime - \textbf{x}\right)
\label{eq:SPH_kernel_delta}
\end{equation}
So it can be viewed as an approximate form of the Dirac function and should satisfy the normalization condition:
\begin{equation}
\int	 w\left(\textbf{x}-\textbf{x}\prime, h\right) d\textbf{x}\prime = \int	 \delta \left(\textbf{x} \prime - \textbf{x}\right) d\textbf{x} \prime =1
\label{eq:SPH-kernel-normalization-prop}
\end{equation}
Besides normalization condition, the weighting function of particle $a$ has to be symmetric with respect to $a$ to ensure that neighbor particles located at the same distance away from $a$ contribute equally to SPH summation equation, see Eq. (\ref{eq:SPH-kernel-symmetric}).
\begin{equation}
w\left(\textbf{x}- \textbf{x} \prime, h\right) = w\left(\textbf{x} \prime - \textbf{x}, h\right)
\label{eq:SPH-kernel-symmetric}
\end{equation}
The weighting function also needs to satisfy conditions such as positivity and compact support. In addition, the kernel function must be monotonically decreasing with the distance between particles.

There is a wide variety of possible weighting functions that can satisfy these requirements, such as spline functions (with different orders) and Gaussian functions. Generally, the accuracy increases with the order of the polynomials of the kernel function, but the computational cost also increases as number of interactions increase. 
We are adopting a truncated Gaussian function as the weighting function in our simulation.
\begin{equation}
w\left(\textbf{x} - \textbf{x} \prime \right) = 
\begin{cases} 
      \dfrac{1}{\left(h \sqrt{\pi}\right)^d} exp \left[- \left(\dfrac{\textbf{x} - \textbf{x} \prime}{h} \right)^2 \right] &  \vert \textbf{x} - \textbf{x} \prime \vert \leq 3h\\
      0 & \text{Otherwise}
\end{cases}
\label{eq:SPH-kernel}
\end{equation}
where $d$ is number of dimensions.
The derivative of the weighting function is:
\begin{equation}
\nabla w\left(\textbf{x} - \textbf{x} \prime \right) = 
\begin{cases} 
      -2\left(\dfrac{\textbf{x} - \textbf{x} \prime}{h}\right) \dfrac{1}{\left(h \sqrt{\pi}\right)^d} exp \left[- \left(\dfrac{\textbf{x} - \textbf{x} \prime}{h}\right)^2 \right] &  \vert \textbf{x} - \textbf{x} \prime \vert \leq 3h\\
      0 & \text{Otherwise}
\end{cases}
\label{eq:SPH-kernel-gradient}
\end{equation}

By replacing $\delta$ function in Eq. (\ref{eq:SPH_kernel_delta}) with the kernel function $w$, an arbitrary function $A\left(\textbf{x}\right)$ can then be approximated by:
\begin{equation}
A\left(\textbf{x}\right) \approx <A\left(\textbf{x}\right)> = \int_{\Omega} A\left(\textbf{x} \prime\right) w\left(\textbf{x}-\textbf{x}\prime, h\right) d\textbf{x}\prime + O\left(h^2\right)
\label{eq:SPH-fundamental-principle}
\end{equation}
As the weighting function is symmetric (Eq. (\ref{eq:SPH-kernel-symmetric})) and satisfies the normalization condition (Eq. (\ref{eq:SPH-kernel-normalization-prop})), odd error terms in Eq. (\ref{eq:SPH-fundamental-principle}) vanish leading to a second order approximation. However, in practice, second order of accuracy can not be achieved because there is no guarantee on the symmetry of particle distribution in real simulation \citep{price2012smoothed}.
Recall that $d\textbf{x}\prime = \dfrac{dm (\textbf{x} \prime)}{\rho (\textbf{x} \prime)}$, the integration equation, Eq. (\ref{eq:SPH-fundamental-principle}), can be approximated by summation and lead to an approximation of the function $A$:
\begin{equation}
<A\left(\textbf{x}\right)> \approx \sum_b m_b \dfrac{A_b}{\rho_b} w\left(\textbf{x}-\textbf{x}_b, h\right)
\label{eq:SPH-approximation-sum}
\end{equation}
where the summation is over all the particles within the region of compact support (See Eq. (\ref{eq:SPH-kernel})) of the weighting function. 
Gradient terms may be straightforwardly calculated by taking the derivative of Eq. (\ref{eq:SPH-approximation-sum}), giving
\begin{equation}
\begin{split}
<\nabla A\left(\textbf{x}\right)> & = \dfrac{\partial }{\partial \textbf{x}} \int_{\Omega} A\left(\textbf{x} \prime\right) w\left(\textbf{x}-\textbf{x}\prime, h\right) d\textbf{x}\prime + O\left(h^2\right) \\
& \approx \sum_b m_b \dfrac{A_b}{\rho_b} \nabla w\left(\textbf{x} - \textbf{x}_b, h\right)
\end{split} 
\label{eq:SPH-scalar-function-gradient}
\end{equation}
For vector quantities the expressions are similar, simply replacing $A$ with $\textbf{A}$ in Eq. (\ref{eq:SPH-approximation-sum}) and Eq. (\ref{eq:SPH-scalar-function-gradient}), giving
\begin{align}
<\textbf{A}\left(\textbf{x}\right)> \approx \sum_b m_b \dfrac{\textbf{A}_b}{\rho_b} w\left(\textbf{x}-\textbf{x}_b, h\right) \\
<\nabla \cdot \textbf{A}\left(\textbf{x}\right)> \approx \sum_b m_b \dfrac{\textbf{A}_b}{\rho_b} \cdot \nabla w\left(\textbf{x} - \textbf{x}_b, h\right) \\
<\nabla \times \textbf{A}\left(\textbf{x}\right)> \approx \sum_b m_b \dfrac{\textbf{A}_b}{\rho_b} \times \nabla w\left(\textbf{x} - \textbf{x}_b, h\right) \\
<\nabla^j \textbf{A}^i\left(\textbf{x}\right)> \approx \sum_b m_b \dfrac{\textbf{A}_b^i}{\rho_b} \nabla^j w\left(\textbf{x} - \textbf{x}_b, h\right) 
\label{eq:SPH-vecctor-function}
\end{align}

\section{Adaptive smoothing length}
The spatial resolution of SPH is determined by the smoothing length, $h$. It is common for each particle to have its own time dependent smoothing length which adapts according to the local number density of particles. Adaptive smoothing length can automatically match the resolution length $h$ to the scale of the system and is especially necessary in certain scenarios where density varies in a large range, for example, Sj$\ddot{o}$green tests.
The adaptivity of smoothing length is less important for incompressible flow while pretty necessary for compressible flow. There are two different ways to adjust the smoothing length.

One robust way of specifying the smoothing length $h$ is proposed by \citet{gingold1978binary}. He suggested that changing of the smoothing length should be related to density according to: 
\begin{equation}
h_a = \sigma \left(\frac{m_a}{\rho_a}\right)^{\frac{1}{d}}
\label{eq:adptive-sml-Gingold}
\end{equation}
Where $d$ is number of dimensions and he suggested to use 1.3 for $\sigma$.
There are many variations of this formulation. For example, \citet{steinmetz1993capabilities} suggested to use averaged local density to determine the smoothing length. \citet{hernquist1989treesph} suggested to keep number of neighbour particles constant by adjusting smoothing length. \citet{inutsuka2002reformulation} suggested to use a more smooth density to avoid large varying of smoothing length within the neighborhood of each particle.
Equation (\ref{eq:adptive-sml-Gingold}) is a nonlinear equation for the single variable $\rho_a$  which can be solved rapidly by iteration, possibly combined with a Newton-Raphson scheme.

The introduction of time varying smoothing lengths can, however, lead to serious problems with and momentum energy conservation in certain situations \citep{hernquist1993some}. Few attention has been paid to losing of strict conservation feature in original SPH discretization when smoothing length varies in time and space. The problem arises from the fact that the use of variable smoothing lengths means that additional terms \citep{nelson1994variable} should appear in the particle equations of motion. In our implementation, we adopt a simple strategy \citep{evrard1988beyond}: $h$ takes the average of $h_a$ and $h_b$. Such strategy can help restore the strict conservation feature of classical SPH discretization providing that the two paired particles are in ``pool of neighbouring particles" of each other. 

\section{Artificial viscosity} \label{sec:artificial-viscosity}
In classical SPH, shock waves are handled by introducing artificial viscosity, a term that is defined based on second derivatives of velocity, to smear out discontinuities and stabilize the simulation. As in the case of first order derivatives, second order derivatives can be estimated by differentiating a SPH interpolation twice. However, such a formulation has two disadvantages: first, it is very sensitive to irregular distribution of particles, second, the second derivative of the kernel can change sign and lead to unphysical representations (for example, viscous dissipation causes decrease of the entropy). 

One of the most commonly used models of artificial viscosity \citep{monaghan1983shock} is:
\begin{equation}
\Pi_{ab}=- \frac{\nu}{\bar{\rho}_{ab}} \dfrac{ \textbf{v}_{ab} \cdot \textbf{x}_{ab}}{\textbf{x}_{ab}^2 + \left(\eta h\right)^2}
\label{eq:art-vis-original}
\end{equation}
The coefficient $\nu$ is defined as:
\begin{equation}
\nu = \alpha \bar{h}_{ab} \bar{c}_{ab}
\end{equation}
where 
\begin{align}
\bar{c}_{ab} = \dfrac{c_a + c_b}{2} \\
\bar{\rho}_{ab} = \dfrac{\rho_a + \rho_b}{2} \\
\textbf{v}_{ab}=\textbf{v}_a-\textbf{v}_b \\
\textbf{x}_{ab}=\textbf{x}_a-\textbf{x}_b
\end{align}
The artificial viscosity term $\Pi_{ab}$ is a Galilean invariant and vanishes for rigid rotation. It produces a repulsive force between two particles when they are approaching each other and an attractive force when they are receding from each other. 

The SPH viscosity can be related to a continuum viscosity by converting the summation to integrals \citep{monaghan2005smoothed}. It has been shown that shear viscosity coefficient $\iota= \frac{\rho \alpha h c}{8} $ and bulk viscosity coefficient $ \zeta = \frac{5 \iota}{3}$ are appropriate for two dimensional flows and modified to $\iota= \frac{\rho \alpha h c}{10} $ , $ \zeta = \frac{5 \iota}{3}$ for three dimensional flows.
An extra term was added to $\nu$ considering aspects of the dissipative term in shock solutions based on Riemann solvers and lead to a new formulation of artificial viscosity. We adopt this new formulation in our simulation:
\begin{equation}
\Pi_{ab}^{\beta} = 
\begin{cases} 
      \dfrac{- \alpha \mu_{ab} \bar{c}_{ab} + \beta \mu_{ab}^2} {\bar{\rho}_{ab}} & \textbf{v}_{ab} \cdot \textbf{x}_{ab} < 0\\
      0 & \textbf{v}_{ab} \cdot \textbf{x}_{ab} > 0
\end{cases}
\label{eq:art-vis-shock}
\end{equation}
where
\begin{equation}
\mu_{ab} = \dfrac{h \textbf{v}_{ab} \cdot \textbf{x}_{ab}}{\textbf{x}_{ab}^2 + \left(\eta h\right)^2} 
\end{equation}
%$\alpha$ and $\beta$ are two parameters that are free to be adjusted for each case. 
$\alpha$ and $\beta$ are two parameters that can be adjusted for different cases.
$\alpha = 1$ and $\beta = 2$ are  recommended by Monaghan for best results. In our simulation, these two parameters are calibrated to  $\alpha = 0.3$ and $\beta = 0.6$. $\eta$ is usually taken as 0.1 to prevent singularities.
%when $\textbf{x}_{ab} = 0$.

\section{Discretization of Euler equations and extensibility}
The basic interpolation given in Eq. (\ref{eq:SPH-approximation-sum}) to Eq. (\ref{eq:SPH-vecctor-function}) provides a general way to obtain SPH expressions of governing equations. The problem is that using these expressions ``{as is}" in general leads to quite poor gradient estimates. Various tricks can be used to conserve linear and angular momentum and thermal energy \citep{monaghan1992smoothed}. Special treatments are also needed for second order derivative terms \citep{monaghan2005smoothed}. We only refer here to one of these possible discretization forms of compressible Euler equations with SPH:
\begin{align}
<\rho_a> = \sum_b m_b w_{ab} \left(h\right) \label{eq:ns-sph-d} \\
\left\langle\dfrac{d \textbf{v}_a}{d t}\right\rangle = -\sum_b m_b \left(\dfrac{p_b}{\rho_b^2} + \dfrac{p_a}{\rho_a^2} + \Pi_{ab}\right) \nabla_a w_{a b}\left(h\right) +\textbf{g} \label{eq:ns-sph-v} \\
\left\langle\dfrac{d e_a}{d t}\right\rangle=
 0.5\sum_b m_b \textbf{v}_{a b}\left(\dfrac{p_b}{\rho_b^2} + \dfrac{p_a}{\rho_a^2} + \Pi_{ab}\right) \cdot \nabla_a w_{a b}\left(h\right) \label{eq:ns-sph-e}
\end{align}
where, $a$ is the SPH particle index. $\Pi$ is an artificial viscosity term, which is discussed in section \ref{sec:artificial-viscosity}. $w_{a b}\left(h\right)$ is a concise form of $w\left(\textbf{x}_a - \textbf{x}_b, h\right)$ and from here on, we will use this concise form.
%However, these discretized formulations will not guarantee conservation of entropy. We can obtain conservation of entropy by giving up on conservation of mass or thermal energy adopting alternative discretized formulations \citep{monaghan1992smoothed}.
As a Lagrangian method, particle position is also updated at every time step.
\begin{equation}
\left\langle\dfrac{d \textbf{x}_a}{dt}\right\rangle = \textbf{v}_a \label{eq:SPH-update-pos}
\end{equation}

We highlight an important feature of the SPH methodology. Adding new physics and new phases into the model is trivial in terms of discretization. For example, adding of new source (or sink) into Eq. (\ref{eq:ns-sph-d}), adding a drag force into Eq. (\ref{eq:ns-sph-v})  and adding a heat exchange term into Eq. (\ref{eq:ns-sph-e}) leads to the new discretization form:
\begin{align}
<\rho_a> = \sum m_b w_{ab} \left(h\right) + \dot{\rho}\left(\textbf{x},t\right)\label{eq:ns-source-sph-d} \\
\left\langle\dfrac{d \textbf{v}_a}{d t}\right\rangle= -\sum_b m_b \left(\dfrac{p_b}{\rho_b^2} + \dfrac{p_a}{\rho_a^2} + \Pi_{ab}\right) \nabla_a w_{a b}\left(h\right) +\textbf{g} + D \sum	_b m_b \dfrac{\textbf{v}_b - \textbf{v}_a}{\rho_b} \label{eq:ns-drag-sph-v}
\end{align}
\begin{equation}
\begin{split}
\left\langle\dfrac{d e_a}{d t}\right\rangle
&= 0.5\sum_b m_b \textbf{v}_{a b}\left(\dfrac{p_b}{\rho_b^2} + \dfrac{p_a}{\rho_a^2} + \Pi_{ab}\right) \cdot \nabla_a w_{a b}\left(h\right) \\
&+ \sum_b \dfrac{m_b}{\rho_b}\left(\kappa_a + \kappa_b\right) \dfrac{\left(T_a - T_b\right)}{\textbf{r}_a - \textbf{r}_b} w_{ab}\left(h\right) \label{eq:ns-conduction-sph-e}
\end{split}
\end{equation}

where the source term $\dot{\rho}$ can be a "sink" of erupted vapor due to its phase change.
%(see microphysics modelling in ATHAM \citep{oberhuber1998volcanic}).
$D$ is a drag force coefficient. $\kappa$ is the heat conduction coefficient. $T$ is the temperature. Other physics can be added easily in a similar way. Adding of these new terms leads to modification of only a few lines in the source code. The drag force term should show up when dynamic disequilibrium between different phases is considered. In that case, each phase needs one set of governing equations of Navier-Stokes type. Adding of new phase into SPH code only needs adding few new lines for the new phase besides interaction terms introduced by the new phase.

\section{Time step}
The physical quantities (velocity, density and pressure) and particle position change every time step. The Courant condition, which is in spirit similar to the Courant condition for the mesh-based methods, is used to determine the time step $\Delta t$.
\begin{equation}
\Delta t = \textrm{CFL} \min_a \bigg \lbrace \dfrac{\left[\frac{m_a}{\rho_a}\right]^{\frac{1}{d}}}{c_a} \bigg \rbrace
\end{equation}
where $c_a$ is sound speed at particle $a$ and calculated based on heat specific ration of the mixture $\gamma_m$ (See Eq. (\ref{eq:sound-speed-mixture})). 
$d$ is number of dimensions. 
\begin{equation}
c_a = \sqrt{\gamma_m \frac{p_a}{\rho_a}}
\label{eq:sound-speed-mixture}
\end{equation}

First order Euler integration, with $\textrm{CFL} = 0.2$, is used to advance in time.

\section{Tensile instability and corrected derivatives}
The classical SPH method was known to suffer from tensile instability and boundary deficiency. Tests of the standard SPH method indicate an instability in the tensile regime, while the calculations are stable in compression.  A simple example of such a test calculation exhibiting the instability involves a body which is subjected to an uniform initial stress, either compressive or tensile. If the initial stress is tensile, a very small velocity perturbation on a single particle can lead to particles clumping together, forming large voids and seriously corrupting density distribution. But if the initial stress is compressive, the small velocity perturbation on a single particle can not lead to any changes in particle distribution \citep{swegle1995smoothed}. To address these difficulties, \citep{chen1999improvement} proposed a corrected SPH formulation. For 1D case, employing a Taylor expansion for $A\left(x\right)$ about $x_a$, multiplying both sides by kernel function and then doing integration over the domain gives
\begin{equation}
\begin{split}
\int_{\Omega} A\left(x\right) w\left(x- x_a, h\right) dx 
& = A_a \int_{\Omega} w\left(x - x_a, h\right) dx \\
&+\dfrac{\partial A}{\partial x}(x_a) \int_{\Omega} \left(x-x_a\right) w\left(x - x_a, h\right) dx +...
\end{split}
\end{equation}
Ignoring derivative terms higher than first order, and writing the integral in particle approximation form leads to:
\begin{equation}
A_a = \frac{\sum_b m_b \dfrac{A_b}{\rho_b} w\left(x_a-x_b, h\right)}{\sum_b m_b \dfrac{1}{\rho_b} w\left(x_a-x_b, h\right)}
\label{eq:CSP-function-approximation-1d}
\end{equation}
Notice that the denominator in Eq. (\ref{eq:CSP-function-approximation-1d}) is actually summation approximation of Eq. (\ref{eq:SPH-kernel-normalization-prop}). That is to say, Eq. (\ref{eq:CSP-function-approximation-1d}) and Eq. (\ref{eq:SPH-approximation-sum}) are the same for particles far away from boundaries as the denominator in Eq. (\ref{eq:CSP-function-approximation-1d}) becomes $1$ in that case.
The first order derivative term can be obtained in a similar way:
\begin{equation}
\nabla A_a = \frac{\sum_b m_b \dfrac{A_b - A_a}{\rho_b} \nabla_a w\left(x_a-x_b, h\right)}{\sum_b m_b \dfrac{x_b - x_a}{\rho_b} \nabla_a w\left(x_a-x_b, h\right)}
\end{equation}

For problems of higher dimension, the expressions for function approximation are exactly the same as Eq. (\ref{eq:CSP-function-approximation-1d}), even though the derivation is different. The first order derivative can be obtained by solving system of equations explicitly or numerically \citep{chen1999improvement}.

\section{Mass fraction update}
Air and erupted material are represented by two different sets of SPH particles (or discretization points) in the model. Based on assumptions we made in section \ref{sec:chp2-Mathematical-Description}, only density is different on two sides of the interface and needs to be updated respectively for each phase. Such assumption also eliminates the surface tension forces across the interface reducing numerical challenges. The updating of density is exactly the same as Eq. (\ref{eq:ns-sph-d}) in spirit. Particles of phase 1 are not counted while evaluating density of phase 2 and vice versa. Updating of density is then based on the following discretized equations.
\begin{equation}
<\rho_{\alpha}^a>=\frac{\sum m_b w_{\alpha b} \left(h\right)}{\sum \frac{m_b}{\rho_b} w_{\alpha b} \left(h\right) +\sum \frac{m_j}{\rho_j} w_{\alpha j} \left(h\right)} \label{eq:gov-sph-d1}
\end{equation}
\begin{equation}
<\rho_\alpha^{sg}>=\frac{\sum_j m_j w_{\alpha j} \left(h\right)}{\sum \frac{m_b}{\rho_b} w_{\alpha b} \left(h\right) +\sum \frac{m_j}{\rho_j} w_{\alpha j} \left(h\right)} \label{eq:gov-sph-d2}
\end{equation}
where the subscript $a$ and $b$ represents air particles (phase 1) while $i$ and $j$ represents erupted material. $\beta$ and $\alpha$ represents either erupted material or air.
$\rho_a^a$ is density of phase 1 (air). 
 $\rho_i^{sg}$ is density of phase 2 (erupted material).
$\rho=\rho^a + \rho^{sg}$ is density of mixture of air and erupted material. 
By definition, the mass fraction is updated according to Eq. (\ref{eq:gov-sph-xi}).
\begin{equation}
<\xi_{\alpha}> = \dfrac{\rho^{sg}_{\alpha}}{\rho_{\alpha}}
\label{eq:gov-sph-xi}
\end{equation}

In areas far away from the interface, updating of density is exactly the same as that for single-phase flow. For example, in the right side and left side (or blue areas) in Fig. \ref{fig:SPH-multiple-density}, where there are only air particles, Eq.  (\ref{eq:gov-sph-d2}) evaluates to zero and total density is:
\begin{equation}
\rho_{\alpha}=\rho_{\alpha}^a=\frac{\sum m_b w_{\alpha b} \left(h\right)}{\sum \frac{m_b}{\rho_b} w_{\alpha b}}
\end{equation}
Which is a special case of Eq. (\ref{eq:CSP-function-approximation-1d}). For these areas occupied by only particles of phase 2 and far away from the interface, similarly, the equation for density update becomes: 
\begin{equation}
\rho_{\alpha}=\rho_{\alpha}^{sg}=\frac{\sum m_j w_{\alpha j} \left(h\right)}{\sum \frac{m_j}{\rho_j} w_{\alpha j}}
\end{equation}
That is to say, the same density updating equation can be applied for both phases and no additional numerical treatment needed to locate where the interface is.

\begin{figure}
\center
\includegraphics[width=12cm]{Chapter-3/Figures/Interface}
\caption{In the left figure, the blue particles (phaseID=1) represent air particles the red ones (phaseID=2) represent erupted material. The right figure shows corresponding mass fraction. Mass fraction are evaluated based on Eq. (\ref{eq:gov-sph-d1}) $\sim$ Eq. (\ref{eq:gov-sph-xi}) without any other interface track or capture method.}
\label{fig:SPH-multiple-density}
\end{figure}

Interface construction will become necessary and important when attempt to include the effects of mixing by resolving the detailed interface structure and dynamics of turbulence. 
As a Lagrangian method, interface tracking in SPH is explicit through capturing of the locations of the particles, much simpler than Euleruan Methods.
The existence of complex evolving interfaces between phases presents severe challenges to conventional Eulerian grid-based numerical methods. Either interface tracking (Lagrangian) \citep{harlow1965numerical, wrobel1991computational, cheng1995simplified} or interface capturing (Eulerian) \citep{hirt1981volume, youngs1982time, gerlach2006comparison, gopala2008volume} methods are used to reconstruct the flow interface of free boundary flow. High computational cost, a tendency to form numerical instabilities and the inability to track complex topological changes are the significant drawbacks of tracking techniques \citep{hirt1981volume, unverdi1992front, anderson1998diffuse}. For interface capturing (Eulerian) method, the surrendering of surface detail before the phase transport calculation means that interface reconstruction is required between time steps to recover the interface information, which needs additional numerical effort \citep{hirt1981volume, youngs1982time}.
Since SPH is able to adaptively adjust the discretization and automatically construct the interface. SPH requires less extra numerical effort for interface construction and therefore is more suitable for volcanic plume simulation.
 
%SPH is able to handle multiphase problems with less additional numerical effort for dealing with interfaces by simply tagging particles of each phase.

\section{Turbulence modeling with SPH}
Turbulence model is essential for both mixing and turbulent momentum and energy exchange. We adopt a LANS type turbulence model, the $SPH-\varepsilon$ turbulence model \citep{monaghan2011turbulence}, which was proposed only for incompressible flow. In the following section, we will extend it for compressible flow. It is necessary to mention that all other existing SPH-SPS turbulence models \citep{holm1999fluctuation, monaghan2002sph, violeau2007numerical} also only focus on incompressible flow.

\subsection{Langrangian average in $SPH-\varepsilon$}
\citet{monaghan2011turbulence} constructed $SPH-\varepsilon$ turbulence model within the framework of SPH in such a way that general principles such as conservation of energy, momentum and circulation are satisfied. The basic idea of $SPH-\varepsilon$ is to determine a smoothed (averaged in space) velocity $\widehat{\textbf{v}}$ by a linear operation on the unsmoothed velocity $\textbf{v}$. The SPH particles move with this smoothed velocity and hence the average motion of the fluid is determined by the averaged velocity $\widehat{\textbf{v}}$:
\begin{equation}
\dfrac{d \textbf{x}_a}{dt} = \widehat{\textbf{v}}_a \label{eq:gov-update-pos-turbulence}
\end{equation}

Average of physical quantities over space introduces extra terms into the governing equations. Once the form of the smoothing (average) is chosen these extra terms are determined.
The typical LANS model uses a smoothed velocity $\widehat{\textbf{v}}$ 
defined in terms of the unsmoothed velocity $\textbf{v}$ by:
\begin{equation}
\widehat{\textbf{v}}\left(\textbf{x}\right)=\int \textbf{v}\left(\textbf{x} \prime\right)G\left(\vert \textbf{x} \prime - \textbf{x} \vert, l\right) d\textbf{x} \prime
\label{eq:turbulence-v-filter}
\end{equation}
where $G$ satisfies:
\begin{equation}
\int G\left(\vert \textbf{x} \prime - \textbf{x} \vert, l\right) d\textbf{x} \prime =1
\end{equation}
and is a member of a sequence of functions which tends to the $\delta$ function in the limit when $ l\rightarrow 0$. A typical example is Gaussian.
The length scale $l$ determines the characteristic width of the kernel and the distance over which the velocity is smoothed.

It is a common practice in LANS to use a differential equation for the smoothing rather than the integral form and finally reach to a system of equations that need to be solved implicitly. In $SPH-\varepsilon$ method, a XSPH \citep{monaghan1989problem} smoothing is adopted which conserves linear and angular momentum. In this way, solving of a system of equations is avoided and it also makes the method simple to implement and cheap for computation. 
The discretized form of the momentum equation is obtained through lengthy derivation. Derivation and other discussions are available in the literature -- see for e.g. \citep{monaghan2011turbulence}. Here we provide a brief summary of key steps.

The smoothing adopted by \citet{monaghan2011turbulence}  is:
\begin{equation}
\widehat{\textbf{v}}\left(\textbf{x}\right)=\textbf{v}\left(\textbf{x}\right)+ \epsilon \int \left(\textbf{v}\left(\textbf{x} \prime\right)-\textbf{v}\left(\textbf{x}\right)\right)G\left(\vert \textbf{x} \prime - \textbf{x} \vert, l\right) d\textbf{x} \prime
\end{equation}
As function $G$ has the same feature as kernel function $w$, SPH approximation of the integration leads to:
\begin{equation} \label{eq:SPH-epsilon-filtering}
\widehat{\textbf{v}}\left(\textbf{x}\right)=\textbf{v}\left(\textbf{x}\right)+\epsilon \sum_b m_b \dfrac{\left(\textbf{v}_b -\textbf{v}\right)}{\rho _b} G\left(\vert \textbf{x} _b - \textbf{x} \vert, l\right)
\end{equation}
By making the replacement:
\begin{equation}
\label{eq:replacement-in-turb-derive}
\dfrac{G\left(\vert \textbf{x} _b - \textbf{x} _a \vert, l\right)}{\rho _b} \rightarrow \dfrac{K_{ab}}{M}
\end{equation}
where $K_{ab} = l^d G_{ab}$, $M = \rho_0 l^d$ in which $d$ is the dimension and $\rho_0$ is initial density. $SPH-\varepsilon$ turbulence model is obtained after lengthy derivation:
\begin{equation}
\label{eq:monaghan-mom-turb}
\dfrac{d \textbf{v}_a}{dt} = -\sum_b \left[ m_b \left(\dfrac{p_b}{\rho_b^2} + \dfrac{p_a}{\rho_a^2}\right) \bigtriangledown_aw_{a b}\left(h\right)\right] + \sum_b m_b \dfrac{\varepsilon}{2} \dfrac{\textbf{v}_{ab} \cdot \textbf{v}_{ab}}{M} \bigtriangledown_a K_{ab}
\end{equation}
Notice that if $l$ is constant: 
%Please notice that the above equation is slightly different from the original equation in Monaghan's paper (Eq. (2.17)). The original equation is based on the assumption that density is uniform initially. This assumption is not general and atmosphere density is stratified in our plume simulation. And the slight modification will not influence following derivation if $l$ is uniform and keep constant because:
%
%{\bf WE NEED TO SAY SOMETHING ABOUT INCOMPRESSIBILITY HERE}
%
\begin{equation}
\nabla K_{ab} = \nabla \left(l^d G_{ab}\right) = l^d \nabla G_{ab}
\end{equation}
The discretized momentum equation with $SPH-\varepsilon$ turbulence model can be written in terms of $G_{ab}$ instead of $K_{ab}$:
\begin{equation}
\label{eq:SPH-mom-epsilon-turb}
\dfrac{d \textbf{v}_a}{dt} = -\sum_b \left[m_b \left(\dfrac{p_b}{\rho_b^2} + \dfrac{p_a}{\rho_a^2}\right) \bigtriangledown_aw_{a b}\left(h\right)\right] + \sum_b m_b \Phi_{ab}\bigtriangledown_aG_{ab}\left(l\right)
\end{equation}
where 
\begin{equation}
\Phi_{ab}=\dfrac{\varepsilon}{2} \dfrac{\textbf{v}_{ab} \cdot \textbf{v}_{ab}}{\rho_b} 
\end{equation}
Which is the extra stress term induced by average. We take coefficient $\varepsilon$ as 0.8 following \citet{monaghan2011turbulence}.

For compressible flow, the energy equation is coupled with the momentum equation and mass conservation equation. Averaging of thermal energy over space introduces some additional terms besides the stress term induced by velocity average \citep{NASACompressibleTurbulence}. The averaged momentum equation for compressible flow are in the same form as that for incompressible flow, all of other additional terms, besides the corresponding velocity average induced stress term, show up in the energy equation. Most turbulence modeling focuses on the stress terms induced by average of velocity. These stress terms are usually either solved directly (for example, LANS methods) or defined via a constitutive relation (for example, large eddy simulation method). Less attention is typically given to the other terms. Most commonly, a Reynolds analogy is used to model the turbulent exchange. Simulations of heat transfer, or other scalar transfer, in turbulent flow simply involve adding transport terms for thermal energy or species concentration, at the expense of greater storage and longer computing times but without other difficulties \citep{cebeci2013analysis}. We adopt this strategy. %PDAC \citep{neri2003multiparticle} also adopted a Reynolds analogy in its turbulence modeling.
The additional terms associated with molecular diffusion and turbulent transport in the energy equation are either modeled in different ways or neglected sometimes \citep{NASACompressibleTurbulence}. We neglect these terms in our simulation.

\subsection{Turbulent heat transfer}
Reynolds analogy is adopted to get the heat transfer coefficient due to turbulence.
The Prandtl number is defined as:
\begin{equation}
Pr=\dfrac{C_p \mu}{\kappa}
\end{equation}
where, $\mu$ is the dynamic viscosity, $\kappa$ is the thermal conductivity. And $\mu$  can be written in term of the absolute viscosity (kinematic viscosity) as:
\begin{align}
\mu=\rho \nu
\end{align}
Then
\begin{equation}
\kappa=\dfrac{C_p \mu}{Pr}
\end{equation}
Typical value of $Pr_t$ for air is 0.7 $\sim$ 0.9 . We take $Pr_t=0.85$ for gases as recommended by \citet{kays1994turbulent} from summarizing of experimental results. 

\citet{monaghan2005smoothed} summarized the simulation of viscosity and heat conduction in his review on SPH. We will refer to his summary in our following discussion. The additional term in discretized momentum equation, Eq. (\ref{eq:SPH-mom-epsilon-turb}), is the turbulent shear stress term. 
Recall that molecular viscosity can be discretized with SPH as shown in Eq. (\ref{eq:art-vis-original}). 
It has been shown that the discretized molecular viscosity has both bulk viscosity and shear viscosity, where shear viscosity coefficient is \citep{monaghan2005smoothed}:
\begin{equation}
\nu_t = S \nu
\end{equation}
with
\begin{equation}
S= 
\begin{cases} 
      \dfrac{1}{10} & if  \quad d=3 \\
      \\
     \dfrac{1}{8}  & if  \quad d=2 
\end{cases}
\end{equation}
The turbulent viscosity coefficient can be inferred from that formulation if we can reformulate the turbulent shear stress term in a form which is similar to the molecular shear term.
Reformulating the turbulent shear stress term:
\begin{equation}
\label{eq:turb-stress-reformulate-to-artificial-vis}
 \sum_b \dfrac{\varepsilon}{2} \dfrac{m_b}{\rho_b} \textbf{v}_{ab} \cdot \textbf{v}_{ab} \nabla_a G_{ab}\left(l_a\right)= \sum_b \dfrac{\varepsilon}{2S} m_b \dfrac{\textbf{v}_{ab}}{\rho_b} \dfrac{S \textbf{v}_{ab} \cdot \textbf{x}_{ab}}{x_{ab}^2} \dfrac{x_{ab}^2}{\textbf{x}_{ab}} \nabla_a G_{ab}\left(l_a\right) 
\end{equation}
then the turbulent viscosity coefficient can be inferred from Eq. (\ref{eq:turb-stress-reformulate-to-artificial-vis}).
\begin{equation}
\nu_t = \dfrac{\varepsilon}{2S} \dfrac{\textbf{v}_{ab} \cdot \textbf{x}_{ab}}{\rho_b}
\end{equation}
Please note that turbulent viscosity term has opposite sign with molecular viscosity term in discretized momentum equation and there is a minus sign in the expression of $\Pi_{ab}$, and they cancel out.

However, the above equation is correct only for the 1D situations. For 2D or 3D, it is not easy to get an explicit expression. 
We adopt an alternative way: obtaining a value for each pair of particles instead of persisting on getting an analytical expression. Choosing the smoothing function as the same as SPH kernel and the smoothing length scale $l$ as the same as smoothing length $h$, the ratio between turbulent shear stress and physical shear stress is: 
\begin{equation}
\begin{split}
\Upsilon_{ab} &= \dfrac{\dfrac{\varepsilon}{2} \dfrac{\textbf{v}_{ab} \cdot \textbf{v}_{ab}}{\rho_b}}{\dfrac{S \nu}{\rho_{ab}} \frac{\textbf{v}_{ab} \cdot \textbf{x}_{ab}}{x_{ab}^2 + \eta^2 h_{ab}^2}} \\
 & = \dfrac{\varepsilon \left(x_{ab}^2 + \eta^2 h_{ab}^2\right)}{2 S \nu} \dfrac{\textbf{v}_{ab} \cdot \textbf{v}_{ab}}{\textbf{v}_{ab} \cdot \textbf{x}_{ab}}
\end{split}
\end{equation}
$\Upsilon_{ab}$ is essentially equivalent to the ratio between turbulent viscous effect of particle b on particle a and molecular viscous effect of particle b on particle a. Turbulent viscosity can be easily obtained by:
\begin{equation}
\begin{split}
\nu_{t,ab} &= \nu \Upsilon_{ab} \\
&= \dfrac{\varepsilon \left(x_{ab}^2 + \eta^2 h_{ab}^2\right)}{2 S} \dfrac{\textbf{v}_{ab} \cdot \textbf{v}_{ab}}{\textbf{v}_{ab} \cdot \textbf{x}_{ab}}
\end{split}
\end{equation}
The corresponding turbulent thermal conductivity should be
\begin{equation}
\kappa_{t,ab}=\dfrac{\varepsilon \overline{C_{p,ab}} \overline{\rho_{ab}} \left(x_{ab}^2 + \eta^2 h_{ab}^2\right) \textbf{v}_{ab} \cdot \textbf{v}_{ab}}{2 S Pr_t\textbf{v}_{ab} \cdot \textbf{x}_{ab}}
\end{equation}
$\overline{C_{p,ab}}$ and $\overline{\rho_{ab}}$ are simply the arithmetic means of specific heat and density. The term used to prevent singularity now can be removed. 
\begin{equation}
\kappa_{t,ab}=\dfrac{\varepsilon \overline{C_{p,ab}} \overline{\rho_{ab}} x_{ab}^2 \textbf{v}_{ab} \cdot \textbf{v}_{ab}}{2 S Pr_t\textbf{v}_{ab} \cdot \textbf{x}_{ab} }
\end{equation}
We also need to prevent singularity, so: 
\begin{equation}
\kappa_{t,ab}= 
\begin{cases} 
      0 & if  \quad \textbf{v}_{ab}=0 \quad or \quad \textbf{x}_{ab}=0 \\
      \dfrac{\varepsilon \overline{C_{p,ab}} \overline{\rho_{ab}} x_{ab}^2 \textbf{v}_{ab} \cdot \textbf{v}_{ab}}{2 S Pr_t\textbf{v}_{ab} \cdot \textbf{x}_{ab} } & \text{otherwise}
\end{cases}
\label{eq:SPH-LANS-heat-conductivity}
\end{equation}

Heat conduction equation without source term is:
\begin{equation}
C_p \dfrac{dT}{dt} = \dfrac{1}{\rho} \nabla \left(\kappa \nabla T\right)
\end{equation}
Second spatial derivative can be approximated with SPH by following \citet{monaghan2005smoothed}: 
\begin{equation}
C_p \dfrac{dT}{dt} = \sum_b \dfrac{m_b}{\rho_a \rho_b} \left(\kappa_a + \kappa_b\right) \left(T_a - T_b\right) F_{ab} \left(h\right)
\end{equation}
where $F_{ab} (h)$ is short for $F \left( \textbf{x}_a - \textbf{x}_b, h \right)$, whose definition is:
\begin{equation}
F_{ab}(h) \textbf{x}_{ab} = \nabla _a w_{ab}
\end{equation}
$F_{ab}$ is always nonpositive, which guarantees that heat flux flows from hot to cold. 
Plug the turbulent thermal conductivity into the heat conduction equation:
\begin{equation}
\begin{split}
C_p \dfrac{dT}{dt}
& = \sum_b \dfrac{m_b}{\rho_a \rho_b} \left(\kappa_a + \kappa_b\right) \left(T_a - T_b\right) F_{ab} \left(h\right) \\
 &= 2 \sum_b \dfrac{m_b}{\rho_a \rho_b} \dfrac{\overline{C_{p,ab}} \overline{\rho_{ab}} \varepsilon x_{ab}^2 \textbf{v}_{ab} \cdot \textbf{v}_{ab}}{2 Pr_t  S \textbf{v}_{ab} \cdot \textbf{x}_{ab} } \left(T_a - T_b\right) F_{ab} \left(h\right)
\end{split}
\label{eq:turb-LANS-heat-conduct1}
\end{equation}
Notice that the number "2" in the front of Eq. (\ref{eq:turb-LANS-heat-conduct1}) comes from integration approximation of second order derivative \citep {cleary1999conduction}. By further simplification, we get:
\begin{equation}
C_p \dfrac{dT}{dt}
 =\dfrac{\varepsilon}{S  Pr_t}  \sum_b \dfrac{m_b}{\rho_a \rho_b} \dfrac{\overline{C_{p,ab}} \overline{\rho_{ab}} x_{ab}^2 \textbf{v}_{ab} \cdot \textbf{v}_{ab}}{\textbf{v}_{ab} \cdot \textbf{x}_{ab}} \left(T_a - T_b\right) F_{ab} \left(h\right)
\end{equation}

\subsection{Discretized governing equations with $SPH-\varepsilon$ turbulence model}
Plugging in the discretized turbulent stress term and turbulent heat transfer term into the momentum and energy equation, we get new discretized governing equations:
\begin{equation}
\begin{split}
\left\langle\dfrac{d \textbf{v}_{\alpha}}{d t}\right\rangle
& =-\sum_b \left[ m_b \left(\dfrac{p_b}{\rho_b^2} + \dfrac{p_{\alpha}}{\rho_{\alpha}^2} + \Pi_{\alpha b}^{\beta} - \Phi_{\alpha b}\right) \bigtriangledown_{\alpha}w_{\alpha b}\left(h\right)\right] \\
& -\sum_j \left[m_j \left(\dfrac{p_j}{\rho_j^2} + \dfrac{p_{\alpha}}{\rho_{\alpha}^2} + \Pi_{\alpha j}^{\beta} - \Phi_{\alpha j}\right) \bigtriangledown_{\alpha}w_{\alpha j}\left(h\right)\right]
+\textbf{g} \label{eq:gov-sph-v}
\end{split}
\end{equation}
With 
\begin{equation}
\Phi_{\alpha \beta}=\dfrac{\varepsilon}{2} \dfrac{\textbf{v}_{\alpha \beta} \cdot \textbf{v}_{\alpha \beta}} {\rho_{\beta}} 
\end{equation}
\begin{equation}
\begin{split}
\left\langle\dfrac{d e_{\alpha}}{d t}\right\rangle
& = 0.5\sum_b \left[m_b \widehat{\textbf{v}_{\alpha b}}\left(\dfrac{p_b}{\rho_b^2} + \dfrac{p_{\alpha}}{\rho_{\alpha}^2} + \Pi_{\alpha b}^{\beta} - \Phi_{\alpha b}\right) \bigtriangledown_{\alpha}w_{\alpha b}\left(h\right)\right] \\
&+ 2 \sum_b \dfrac{m_b}{\rho_{\alpha} \rho_b} \kappa_{t,\alpha b} \left(T_{\alpha} - T_b\right) F_{\alpha b} \left(h\right) \\
& +0.5\sum_j \left[m_j \widehat{\textbf{v}_{\alpha b}}\left(\dfrac{p_j}{\rho_j^2} + \dfrac{p_{\alpha}}{\rho_{\alpha}^2} + \Pi_{\alpha j}^{\beta} - \Phi_{\alpha j}\right) \bigtriangledown_{\alpha}w_{\alpha j}\left(h\right)\right] \\
&+ 2 \sum_j \dfrac{m_j}{\rho_{\alpha} \rho_j} \kappa_{t,\alpha j} \left(T_{\alpha} - T_j\right) F_{\alpha j} \left(h\right)
\end{split}
\label{eq:gov-sph-e}
\end{equation}
with $\kappa_{t,\alpha \beta}$ given by Eq. (\ref{eq:SPH-LANS-heat-conductivity}). 
As the particle-scale movement of flow is based on smoothed velocity, the velocity in the energy equation should also be smoothed.
The filtering process is done according to Eq. (\ref{eq:SPH-epsilon-filtering}). Position of particles is updated according to Eq. (\ref{eq:gov-update-pos-turbulence}). Smoothed velocity is also used while computing artificial viscosity.
% $\Pi_{ab}$.

\section{Boundary conditions} \label{sec:SPH-bc}
All boundary conditions are imposed by ghost particles. Subfigure $b$ of Fig. \ref{fig:bc_and_domain_decomp} shows how boundaries are deployed. 
%Physical quantities of particles, including real particles and different types of ghost particles, are updated differently.
\subsection{Wall boundary condition}
Traditionally either ghost particles that mirror real particles across the boundary \citep {ferrari2009new} or boundary forces \citep {monaghan2009sph} have been used to impose the wall boundary conditions. One disadvantage of the latter  is that the boundary forces tend to corrupt the solution in the local neighborhood. In addition, a natural way of imposing eruption boundary condition is using eruption ghost particles. To impose boundary conditions in a consistent way, we adopt a modified version of the ghost particle method \citep {kumar2013parallel} for wall boundary conditions. Stationary wall ghost particles are deployed in the same way as real particles. Instead of enforcing symmetry particle by particle, a symmetric field across the boundary is explicitly enforced. Ghost particles are reflected into the domain and physical quantities are calculated at these reflected positions by SPH interpolations. It should be noted that wall ghost particles should not be counted when computing physical properties of these reflected positions. Assign all properties, except for velocity, at the corresponding reflected position to the ghost particle. The velocity of each wall ghost particles is set to have the same value but opposite direction of the interpolated velocity at its corresponding reflection. By this way, the no-slip wall boundary condition (Eq. (\ref{eq:wall_bc_v})) is imposed naturally. These wall ghost particles serve as neighbors in momentum and energy update. More implementation details about this method can be found in \citep {kumar2013parallel}. As these wall ghost particles are stationary, there is no mass flux on the boundary (Eq. (\ref{eq:wall_bc_rho}) and (\ref{eq:wall_bc_xi})). In addition, as temperature is also symmetric with respect to the boundary, the gradient of temperature vanishes and hence there is no internal energy flux on the wall boundary (Eq. (\ref{eq:wall_bc_e})). 
In our current model, the ground is assumed to be  flat. For more complicated topography, it has been shown in other work \citep {kumar2013parallel} that this method also works as well.

\subsection{Eruption boundary condition}
A natural way of imposing eruption boundary condition is using ghost particles that move with the eruption velocity and bear the temperature of the erupted material. A parabolic velocity profile that represents a fully developed Hagen-Poiseuille flow is used to determine the inlet particle velocity in Eq. (\ref{eq:erupt_bc_v}). The detailed shape of the parabolic profile is determined based on the averaged eruption velocity $\bar{\omega}$.
\begin{equation}
\omega (x, y) = 2 \bar{\omega} \left(1-\frac{(x-x_0)^2 + (y-y_0)^2}{r^2}\right)
\label{eq:erupt_bc_v-discrete}
\end{equation}
where $\bar{\omega}$ is the mean eruption velocity, which is estimated based on mass flux rate, radius of the eruption conduit, and density of the erupted material. $(x_0, y_0)$ is the position of the center line of the eruption conduit.

The mass of eruption ghost particles is set to a value so that evaluation of Eq. (\ref{eq:ns-sph-d}) can result in a density that is consistent with the value given in Eq. (\ref{eq:erupt_bc_rho}). 
The internal energy associated with these particles are set to a value so that Eq. (\ref{eq:erupt_bc_e}) is satisfied. The mass fraction of erupted material (Eq. (\ref{eq:erupt_bc_xi})) is automatically satisfied as all particles in the eruption conduit are of phase 2. The density, momentum, and internal energy of these eruption ghost particles are not updated before they move above ground while only position is updated. As soon as they move out from eruption conduit, these ghost particles will be shifted to real particles and from here on their physical quantities and position will be updated based on discretized governing equations. New ghost particles need to be added at the bottom of the eruption conduit as these existing ghost particles move upwards.

\subsection{Pressure boundary condition}
Another boundary condition in our model is the pressure outlet boundary. For flow in a straight channel, it is possible to treat the exit the same as the entry with a prescribed velocity profile. For flow with more complex channel, an exit far downstream of the flow disturbance is also feasible. However, the natural boundary condition (Eq. (\ref{eq:pressure_bc_p})) is more suitable for plume simulation as the outlet is open atmosphere. The way we impose pressure boundary condition is: adding several layers of pressure ghost particles surrounding the real atmosphere particles. Pressure, density and temperature are determined based on the elevation of pressure ghost particles. Velocity is set to zero for static atmosphere. The physical quantities for pressure ghost particles are not updated while these for real particles are updated at every time step. As position of all pressure ghost particles keeps constant, we essentially impose a static pressure boundary condition. Real particles are removed as soon as they move out from the computational domain.

As simulations progress, changes in position and physical quantities of real particles near pressure boundaries might corrupt pressure boundary condition that was established initially. This shortcoming is relieved by choosing a larger computational domain so that boundaries that might be corrupted locate far away from turbulent mixing area. In addition, to avoid enlarging fluctuations, we add another constraint on the time step: 
\begin{equation}
\Delta t \leq CFL_p \dfrac{h}{v}
\end{equation}
where $CFL_p$ is a safety coefficient which has similar function as the normal $CFL$ number. Too small $CFL_p$ would slow down simulation while too large $CFL_p$ would lose its ability of mitigate numerical fluctuation near the boundary. The proper $CFL_p$ is determined by a series of simulation tests.

\begin{figure}
\center
\includegraphics[width=0.99 \textwidth]{Chapter-3/Figures/t120_bc_proc}
\caption{A cross section view of the simulation domain in $y-z$ plane at 66 seconds. Subfigure $a$ shows the mass fraction. Subfigure $b$ shows all boundary conditions: the dark blue region is occupied by eruption ghost particles with "Ghost particle ID" of 0, the light blue area is occupied by pressure ghost particles with "Ghost particle ID" of 1, the gray area is filled with wall ghost particles with "Ghost particle ID" of 2, "Ghost particle ID" of all real particles are set to 100, they occupy the major portion of the domain in subfigure $b$. Subfigure $c$ shows the cross section view of domain decomposition based on SFC. The simulation is conducted on 12 processors, so there are 12 subdomains in total. The cross section view cut through several subdomains and shows a crosssectional view of domain decomposition.}
\label{fig:bc_and_domain_decomp}
\end{figure}